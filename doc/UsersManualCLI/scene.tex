\chapter{シーン} \label{sec:scene} ~

この章ではシーンと呼ばれる概念と、それを FK システム上で
実現した fk\_Scene というクラスに関しての使用法を述べる。

シーンは、複数のモデルとカメラからなる「場面」を意味する。シーンには
複数の描画するためのモデル及びカメラを示すモデルを登録する。
このシーンをウィンドウに設定することによって、
そのシーンに登録されたモデル群が描画される仕組みになっている。

このシーンは、(fk\_Scene クラスのオブジェクト
というかたちで) 複数存在することができる。これは、あらかじめ全く
異なった世界を複数構築しておくことを意味する。状況によって様々な
世界を切替えて表示したい場合には、どのシーンをウィンドウ
に設定するかをうまく選択していけばよい。

\section{モデルの登録}
モデルの登録は、entryModel() というメンバ関数を用いる。これは次のように
使用される。
\\
\begin{breakbox}
\begin{verbatim}
        fk_Model model1, model2, model3;
        fk_Scene scene;

        scene.entryModel(&model1);
        scene.entryModel(&model2);
        scene.entryModel(&model3);
\end{verbatim}
\end{breakbox}
~ \\
モデルを登録する際、引数はポインタとして渡す。
scene オブジェクトの中でこのポインタが参照される。
登録したモデルをリストから削除したい場合は、removeModel() 関数を用いる。
もし、シーン中に登録されていないモデルに対して removeModel() を用いた場合には、
特に何も起こらない。
以下の例は、DrawModeFlag が true の場合はモデルをシーンに登録し、
そうでない場合はモデルをシーンから削除する。
\\
\begin{breakbox}
\begin{verbatim}
        fk_Model model1;
        fk_Scene disp;
        bool DrawModelFlag;

                :
                :

        if(DrawModelFlag == true) {
                scene.entryModel(&model1);
        } else {
                scene.removeModel(&model1);
        }
\end{verbatim}
\end{breakbox}
~ \\
また、一旦シーン中のモデルを全てクリアしたい場合には、
clearModel() という関数を呼ぶことで実現できる。

透明度が設定されているモデルがある場合、シーンへの登録の
順序によって結果が異なることがある。具体的に述べると、透明なモデルの
後に登録されたモデルは、透明なモデルの裏側にあっても表示されなくなる。
これを防ぐには、透明な立体を常にディスプレイリストのできるだけ後ろに
登録しておく必要がある。具体的には、別のモデルを登録した後に
透明なモデルを entryModel() メンバ関数によって再び登録しなおせばよい
\footnote{わざわざ removeModel() を呼ばなくても、自動的に重複した
モデルはシーンから削除されるようになっている。}。

ちなみに、実際に描画の際に透過処理を行うには fk\_Scene オブジェクトに
おいて setBlendStatus() メンバ関数を用いて透過処理の設定を行う必要が
ある。これは第 \ref{sec:scenetrans} 節に詳しく述べる。

\section{カメラ(視点)の設定} \label{sec:scenecamera}
カメラ(視点)の設定は、モデル(のポインタ)を entryCamera() メンバ関数によって
代入することによって行われる。
\\
\begin{breakbox}
\begin{verbatim}
        fk_Model camera;
        fk_Scene scene;

        scene.entryCamera(&camera);
\end{verbatim}
\end{breakbox}
~ \\
entryCamera() によってカメラが登録された場合、カメラを意味するオブジェクトの
位置や方向が変更されれば、再代入する必要なくカメラは変更される。これは
多くの場合都合がよい。もし、別のモデルをカメラとして利用したいのならば、
別のモデルのポインタを代入すればよい。

\section{背景色の設定} \label{sec:scenebg}
FK システムでは、背景色はデフォルトで黒に設定されているが、次のように
fk\_Scene の setBGColor() メンバ関数を用いることで背景色を変更することが
できる。引数としては、RGB 要素を示す 0 〜 1 までの数値3つ、あるいは
fk\_Color 型の変数1つの2種類がある。以下の例は、背景色を青色に
設定している例である。
\\
\begin{breakbox}
\begin{verbatim}
        fk_Scene scene;
               :
               :
        scene.setBGColor(0.0, 0.0, 1.0);
\end{verbatim}
\end{breakbox}

\section{透過処理の設定} \label{sec:scenetrans}
モデルのマテリアルにおいて、setAlpha() メンバ関数を用いて立体に
透明度を設定することが可能であるが、実際に透過処理を行うには
fk\_Scene クラスのオブジェクトにおいて setBlendStatus() メンバ関数を
用いて設定を行わなければならない。具体的には、次のように引数に true を
入力することによって透過処理の設定が行える。
\\
\begin{screen}
\begin{verbatim}
        fk_Scene scene;
                :
                :
        scene.setBlendStatus(true);
\end{verbatim}
\end{screen}
~ \\
透過処理を無効にしたい場合は、false を入力すればよい。もし透過処理を
ON にした場合、(実際に透明なモデルが存在するか否かに関わらず)描画処理が
ある程度低速になる。

\section{霧の効果}
FK システムでは、シーン全体に霧効果を出す機能がサポートされている。
まず、霧効果の典型的な利用法を \ref{sec:scenefogintro} 節で解説し、
霧効果の詳細な利用方法を \ref{sec:scenefogref} 節で述べる。

\subsection{霧効果の典型的な利用方法} \label{sec:scenefogintro}
霧効果は、次の 4 項目を設定することで利用できる。
\begin{itemize}
 \item 減衰関数の設定。
 \item オプションの設定。
 \item 係数の設定。
 \item 霧の色設定。
\end{itemize}
これを全て行うコード例は、以下のようなものである。
\\
\begin{breakbox}
\begin{verbatim}
        fk_Scene        scene;

        scene.setFogMode(FK_LINEAR_FOG);
        scene.setFogOption(FK_FASTEST_FOG);
        scene.setFogLinearMap(0.0, 400.0);
        scene.setFogColor(0.3, 0.4, 1.0, 0.0);
\end{verbatim}
\end{breakbox}
~ \\
まず、setFogMode() によって減衰関数を指定する。通常は、例にあるように
FK\_LINEAR\_FOG を指定すれば良い。

次に、setFogOption() でオプションを指定する。これは \ref{sec:scenefogref} 節
で述べるような3種類があるので、目的に応じて適切に設定する。

次に、setFogLinearMap() によって霧効果の現れる領域を設定する。最初の数値が
霧が出始める距離、後の数値が霧によって何も見えなくなる距離である。

最後に、setFogColor() によって霧の色を設定する。通常は背景色と同一の色を
設定すればよい。また、最後の透過度数値は 0 を指定すれば良い。

以上の項目を設定するだけで、シーンに霧効果を出すことが可能となる。
より詳細な設定が必要な場合は、次節を参照すること。
\subsection{霧効果の詳細な利用方法} \label{sec:scenefogref}
霧効果に関連する機能として、
次のような fk\_Scene のメンバ関数群が提供されている。
\begin{bf}
\begin{tabbing}
xxxxxxx \= \kill
\> void setFogMode(fk\_FogMode); \\
\> void setFogOption(fk\_FogOption); \\
\> void setFogDensity(double); \\
\> void setFogLinearMap(double, double); \\
\> void setFogColor(fk\_Color); \\
\> void setFogColor(float, float, float, float); \\
\> void setFogColor(double, double, double, double); \\
\end{tabbing}
\end{bf}
以下に、各関数の解説を述べる。
\begin{description}
 \item[\hspace*{0.6cm}void setFogMode(fk\_FogMode mode)] ~ \\
	霧による減衰の関数を設定する。以下のような項目が入力できる。
	各数値の設定はその他の設定関数を参照すること。
	\begin{center}
	\begin{tabular}{|c|p{7cm}|}
	\hline
	FK\_LINEAR\_FOG & 減衰関数として \(\dfrac{E - z}{E - S}\) が
		選択される。\\ \hline
	FK\_EXP\_FOG & 減衰関数として \(e^{-dz}\) が選択される。\\ \hline
	FK\_EXP2\_FOG & 減衰関数として \(e^{-(dz)^2}\) が選択される。\\ \hline
	FK\_NONE\_FOG & 霧効果を無効とする。\\ \hline
	\end{tabular}
	\end{center}
	なお、デフォルトでは FK\_NONE\_FOG が選択されている。

 \item[\hspace*{0.6cm}void setFogOption(fk\_FogOption opt)] ~ \\
	霧効果における描画オプションを設定する。以下のような項目が
	入力できる。
	\begin{center}
	\begin{tabular}{|c|p{7cm}|}
	\hline
	FK\_FASTEST\_FOG & 描画の際に、速度を優先する。\\ \hline
	FK\_NICEST\_FOG & 描画の際に、精細さを優先する。\\ \hline
	FK\_NOOPTION\_FOG & 特にオプションを設定しない。\\ \hline
	\end{tabular}
	\end{center}
	なお、デフォルトでは FK\_NOOPTION\_FOG が選択されている。

 \item[\hspace*{0.6cm}void setFogDensity(double d)] ~ \\
	減衰関数として FK\_EXP\_FOG (\(e^{-dz}\)) 及び
	及び FK\_EXP2\_FOG (\(e^{-(dz)^2}\)) が選択された際の、	
	減衰指数係数 \(d\) を設定する。ここで、\(z\) はカメラから
	対象地点への距離を指す。

 \item[\hspace*{0.6cm}void setFogLinearMap(double S, double E)] ~ \\
	減衰関数として FK\_LINEAR\_FOG (\(\dfrac{E - z}{E - S}\)) が
	選択された際の、減衰線形係数 \(S, E\) を設定する。もっと
	平易に述べると、霧効果が現れる最初の距離 \(S\) と、
	霧で何も見えなくなる距離 \(E\) を設定する。

 \item[\hspace*{0.6cm}void setFogColor(fk\_Color col)]
 \item[\hspace*{0.6cm}void setFogColor(float R, float G, float B, float A)]
 \item[\hspace*{0.6cm}void setFogColor(double R, double G, double B, double A)]
	~ \\
	霧の色を設定する。大抵の場合、背景色と同一の色を指定し、
	透過度は 0 にしておく。
\end{description}

\section{オーバーレイモデルの登録}

表示したい要素の中には、物体の前後状態に関係なく常に表示されてほしい場合がある。
例えば、画面内に文字列を表示する場合などが考えられる。このような処理を実現するため、
通常のモデル登録とは別に「オーバーレイモデル」として登録する方法がある。
モデルをオーバーレイモデルとしてシーンに登録した場合、表示される大きさや色などは
通常の場合と変わらないが、描画される際に他の物体よりも後ろに存在していたとしても、
常に全体が表示されることになる。

オーバーレイモデルは一つのシーンに複数登録することが可能である。
その場合、表示は物体の前後状態は関係なく、後に登録したモデルほど前面に
表示されることになる。

基本的には、通常のモデル登録と同様の手順でオーバーレイモデルを登録することができる。
オーバーレイモデルを扱うメンバ関数は、以下の通りである。
\begin{tabbing}
xx \= xxxx \= \kill
\> \textbf{void entryOverlayModel(fk\_Model *model)} \\
	\> \> \begin{tabular}{p{15cm}}
		モデルをオーバーレイモデルとしてシーンに登録する。
		「model」が既にオーバーレイモデルとして登録されていた場合は、
		そのモデルが最前面に移動する。
	\end{tabular} \\ \\

\> \textbf{void removeOverlayModel(fk\_Model *model)} \\
	\> \> \begin{tabular}{p{15cm}}
		「model」がオーバーレイモデルとして登録されていた場合、
		リストから削除する。
	\end{tabular} \\ \\

\> \textbf{void clearOverlayModel(void)} \\
	\> \> \begin{tabular}{p{15cm}}
		全ての登録されているオーバーレイモデルを解除する。
	\end{tabular}
\end{tabbing}


\section{立体視の制御}

FKシステムではOpenGLの拡張機能を利用し、立体視画像を出力することが可能である。
立体視を利用するには「Quad Buffer」と呼ばれる機能をサポートするビデオカードと、
120Hzの高周波数出力が可能なディスプレイ、およびシャッター式の眼鏡が必要である。
これらの環境を満たしてさえいれば、FKシステムを利用しているプログラムに対してわずか
数行を書き足すだけで、立体視画像を出力するプログラムに拡張することができる。

立体視機能を有効にするには、fk\_Windowクラスのオブジェクトに対して
setOGLStereoMode()メンバ関数を呼び出す必要がある。引数にtrueを渡すことで
有効化できるが、前述したハードウェア要件を満たすビデオカードが認識できない場合、
プログラムはその時点で強制終了してしまう。現時点では事前に立体視対応を判定する手段を
FKシステムでは提供していないので注意すること。

fk\_Sceneクラスでは、立体視表現のために左目と右目を表すカメラモデルを
追加で設定することができるようになっている。射影変換の設定もカメラモデルごとに
別個の設定を持つことができるため、両眼視差の厳密な微調整が可能である。
これら立体視表現のためのメンバ関数は以下の通りである。
\begin{tabbing}
xx \= xxxx \= \kill
\> \textbf{void entryStereoCamera(fk\_StereoChannel channel, fk\_Model *model)} \\
	\> \> \begin{tabular}{p{15cm}}
		モデルを左目、あるいは右目のカメラモデルとしてシーンに登録する。
		「channel」には、「FK\_STEREO\_LEFT」か「FK\_STEREO\_RIGHT」の
		どちらかを指定する。
	\end{tabular} \\ \\

\> \textbf{void setStereoProjection(fk\_StereoChannel channel, fk\_Projection *proj)} \\
	\> \> \begin{tabular}{p{15cm}}
		左目、あるいは右目の射影設定を変更する。「channel」の指定は前述の通りである。
		「proj」にはfk\_Perspective、fk\_Ortho、fk\_Frustumクラスの
		オブジェクトのポインタを指定する。詳細はリファレンスマニュアルを参照のこと。
	\end{tabular} \\ \\

\> \textbf{const fk\_Model * getStereoCamera(fk\_StereoChannel channel)} \\
\> \textbf{const fk\_ProjectBase * getStereoProjection(fk\_StereoChannel channel)} \\
	\> \> \begin{tabular}{p{15cm}}
		立体視用に設定したカメラモデル、または射影設定を取得する。
		「channel」の指定は前述の通りである。
	\end{tabular} \\ \\

\> \textbf{void clearStereo(void)} \\
	\> \> \begin{tabular}{p{15cm}}
		全ての立体視関連の設定を初期化する。左右のカメラモデルと射影設定は、
		通常のカメラに設定されているものと同一になる。
	\end{tabular}
\end{tabbing}

左右のカメラに対しては、通常のカメラモデルを親としたモデルを設定すると、
視差を保ったまま通常のカメラの移動や回転が反映されるので便利である。
以下に立体視の設定を施すコーディングの一例を示す。
\\
\begin{breakbox}
\begin{verbatim}
        fk_Window  window;
        fk_Scene   scene;
        fk_Model   camera, camL, camR;

        // 立体視の有効化
        window.setOGLStereoMode(true);
        // 通常のカメラをセット
        scene.entryCamera(&camera);
        // 左右のカメラを通常カメラの子としてセット
        camL.setParent(&camera);
        camR.setParent(&camera);
        // 左右それぞれの位置を中心から左右にずらす
        camL.glTranslate(-10.0, 0.0, 0.0);
        camR.glTranslate( 10.0, 0.0, 0.0);
        // 両眼が同じ場所を注視するように向きをセット
        camL.glFocus(0.0, 0.0, -10.0);
        camR.glFocus(0.0, 0.0, -10.0);
\end{verbatim}
\end{breakbox}
~ \\
以上のように設定した上で通常のカメラを制御すれば、左右のカメラも連動して動作する。
ただし、これだけでは射影設定がデフォルトのままであるため、厳密な立体視表現には
不十分である。より立体感を追求するには、左右のカメラに異なる一般透視投影を用いて
微調整を行う必要がある。立体視の原理、技法に関しては関連書籍を参照すること。
