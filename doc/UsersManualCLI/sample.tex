\chapter{サンプルプログラム} \label{chap:sample}
\section{基本的形状の生成と親子関係}
次に掲載するプログラムは、原点付近に1個の直方体と2本の線分を作成し
表示するプログラムである。ただ表示するだけでは面白くないので、線分を
直方体の子モデルにし、直方体を回転させると線分も一緒に回転することを
試してみる。また、視点も最初は遠方に置いて段々近づけていき、ある程度まで
接近したらひねりを加えてみる。

\begin{center}
詳細解説
\end{center}

\begin{itemize}

\item 2 行目の using 文は、FK システムを使用する場合に必ず記述する必要がある。

\item 11 行目で標準組込マテリアルの初期化を行っている。
      26行目で直方体のマテリアルに組込マテリアルを使用しているが、
      このように組込マテリアルを利用する場合はそれより前に初期化を行う必要がある。

\item 20 行目で FPS (リフレッシュレート) の設定を行っているが、
	FPS プロパティに 0 を代入すると FPS 制御を行わずに最高速で表示する。
	(ただし、環境によっては自動的に FPS 制御が行われる場合がある。)

\item 23 行目以降で直方体の生成を行っている。
      基本的には、モデルと形状を用意し、モデルの Shape プロパティに形状を設定しておき、
      モデルに対して各種の設定を行うというスタイルとなる。
      作成したモデルは 27 行目にあるようにウィンドウに登録を行わないと表示されないので
      注意が必要である。

\item 30 行目では、線分形状の端点となる各点の位置ベクトルを配列として生成している。
	この時点では pos 内の各変数のインスタンスはまだ確保されていないため、
	31,32 行目のように new によってインスタンスを生成する必要があることに注意する。

\item 35,36 行目ではそれぞれ線分形状と線分モデルを配列として生成している。
	これも前述した pos と同様、38 行目や 40 行目にあるようにインスタンスを new で
	用意する必要がある。

\item 42 行目では、lineModel[i] の親モデルを blockModel に設定している。
	親子関係は、このように Parent プロパティに親モデルとなる fk\_Model 型インスタンスを
	設定する方法がもっとも標準的なものである。
	これにより、各線分モデルは直方体モデルの動きに追従するようになる。
	しかし、親子関係は表示属性にはなんら影響はしないため、
	各線分を表示するためには 43 行目にあるようにウィンドウへの登録が必要となる。
	
\item 45、46 行目で、線分に対して色設定を行っている。
	どのような形状であっても、線に対して色を設定する場合は
	LineColor プロパティを用いる。

\item 49 〜 53 行目で、カメラ(視点)モデルの設定を行っている。
	カメラモデルは通常の fk\_Model インスタンスを準備すればよく、
	このサンプルの場合は位置を \((0,0, 2000)\)、方向を原点に向け、
	ウィンドウの上方向が \((0, 1, 0)\) になるように設定している。
	ある fk\_Model 型インスタンスをカメラモデルとして設定するには、
	54 行目にあるように fk\_AppWindow クラスの CameraModel プロパティに
	設定を行えばよい。設定後にカメラモデルを動作させれば視点も動的に変化し、
	任意のタイミングで別のインスタンスに変更することも可能である。
	
\item 58 行目の for 文中にある Update() メソッドは2つの働きがある。
	まず、現在のモデル設定に従って画面の描画を更新する。
	また、そのタイミングでウィンドウが消去されていないかをチェックし、
	もし消去されていた場合は false を返し、正常な場合は true を返す。

	ウィンドウは、画面上で「ESC」ボタンが押されていたり、
	ウィンドウが OS 内の機能で強制的に閉じられた場合に消去される。

 \item 63 行目で、直方体 (と子モデルである線分) を Y 軸中心に
	回転させている。回転角度は、\(\pi/300 = 0.6 ^{\circ}\) である。
	ここにある「Math.PI」は C\# で利用できる円周率を表す。

 \item 65 行目で、常にカメラの注視点が原点に向くように補正を行っている。

 \item 66 行目で、count が 1000 を超えた場合に
	視点にひねりを加えている。
\end{itemize} ~
\begin{breakbox}
\begin{small}
\begin{verbatim}
 1: using System;
 2: using FK_CLI;
 3: 
 4: namespace FK_CLI_Box
 5: {
 6:     class Program
 7:     {
 8:         static void Main(string[] args)
 9:         {
10:             // マテリアル情報の初期化
11:             fk_Material.InitDefault();
12: 
13:             // ウィンドウ生成
14:             var win = new fk_AppWindow();
15: 
16:             // ウィンドウサイズ設定
17:             win.Size = new fk_Dimension(600, 600);
18: 
19:             // Frame Per Seconds (画面更新速度) の設定
20:             win.FPS = 60;
21: 
22:             // 直方体モデル生成
23:             var blockModel = new fk_Model();
24:             var block = new fk_Block(50.0, 70.0, 40.0);
25:             blockModel.Shape = block;
26:             blockModel.Material = fk_Material.Yellow;
27:             win.Entry(blockModel);
28: 
29:             // 線分モデル生成
30:             fk_Vector[] pos = new fk_Vector[4];
31:             pos[0] = new fk_Vector(0.0, 100.0, 0.0);
32:             pos[1] = new fk_Vector(100.0, 0.0, 0.0);
33:             pos[2] = -pos[0];
34:             pos[3] = -pos[1];
35:             fk_Line[] line = new fk_Line[2];
36:             fk_Model[] lineModel = new fk_Model[2];
37:             for(int i = 0; i < 2; i++) {
38:                 line[i] = new fk_Line();
39:                 line[i].PushLine(pos[2*i], pos[2*i + 1]);
40:                 lineModel[i] = new fk_Model();
41:                 lineModel[i].Shape = line[i];
42:                 lineModel[i].Parent = blockModel;
43:                 win.Entry(lineModel[i]);
44:             }
45:             lineModel[0].LineColor = new fk_Color(1.0, 0.0, 0.0);
46:             lineModel[1].LineColor = new fk_Color(0.0, 1.0, 0.0);
47: 
48:             // カメラモデル設定
49:             var camera = new fk_Model();
50:             camera.GlMoveTo(0.0, 0.0, 2000.0);
51:             camera.GlFocus(0.0, 0.0, 0.0);
52:             camera.GlUpvec(0.0, 1.0, 0.0);
53:             win.CameraModel = camera;
54:             win.Open();
55: 
56:             var origin = new fk_Vector(0.0, 0.0, 0.0);
57: 
58:             for(int count = 0; win.Update() == true; count++) {
59:                 // カメラ前進
60:                 camera.GlTranslate(0.0, 0.0, -1.0);
61: 
62:                 // ブロックを y 軸中心に回転
63:                 blockModel.GlRotateWithVec(origin, fk_Axis.Y, Math.PI/300.0);
64: 
65:                 // カメラの注視点を原点に向ける
66:                 camera.GlFocus(origin);
67: 
68:                 // カウンターが1000を上回ったらカメラをz軸中心に回転
69:                 if (count >= 1000) camera.LoRotateWithVec(origin, fk_Axis.Z, Math.PI/500.0);
70:             }
71:         }
72:     }
73: }
\end{verbatim}
\end{small}
\end{breakbox}
\section{形状の簡易表示とアニメーション} \label{sec:sampleviewer}

次のサンプルは、fk\_ShapeViewer クラスの典型的な利用法を示したものである。

\begin{itemize}
 \item 21 〜 27 行目で 11 行 11 列の行列として並んでいる状態の
	座標を計算している。その際、\(z = \dfrac{x^2 - y^2}{10}\)
	として \(z\) 成分は計算されている。

 \item 29 〜 36 行目でインデックスフェースセットを表す
	配列を作成している。インデックスフェースセットに関しては、
	第 \ref{sec:solidGen1} 節を参照すること。

 \item 38 行目で実際に形状を生成する。この部分の解説も、
	第 \ref{sec:solidGen1} 節に記述がある。

 \item 39 行目では、作成した形状を描画形状として登録している。
	今回は登録する形状が一つだけなので Shape プロパティに代入することで登録を
	行っているが、複数の形状を同時に表示したい場合は fk\_ShapeViewer クラスの
	SetShape() メソッドを用いるとよい。

 \item 40 行目では、表裏の両面及び稜線を描画するように設定している。

 \item 43 行目では for ループ中で描画が行われるよう記述されている。
	これにより、44 〜 50 行目が実行される度に描画処理が行われるようになる。

 \item 46 行目では、アニメーションの際の頂点移動量が計算されている。
	移動は \(z\) 方向のみ行われ、移動量は
	\(\sin\frac{counter + 10j}{5\pi}\)である。counter はループの度に
	43 行目で 10 ずつ追加されているので、描画の度に移動量が異なる
	ことになる。

 \item 47 行目で初期位置に移動量が足され、48 行目で実際に各頂点を移動している。
\end{itemize}

\begin{breakbox}
\begin{small}
\begin{verbatim}
 1: using System;
 2: using FK_CLI;
 3: 
 4: namespace FK_CLI_Viewer
 5: {
 6:     class Viewer
 7:     {
 8:         static void Main(string[] args)
 9:         {
10:             const int SIZE = 40;
11: 
12:             var viewer = new fk_ShapeViewer(600, 600);
13:             var shape = new fk_IndexFaceSet();
14:             var pos = new fk_Vector[(SIZE+1)*(SIZE+1)];
15:             var moveVec = new fk_Vector();
16:             var movePos = new fk_Vector();
17:             var IFSet = new int[4*SIZE*SIZE];
18:             int i, j;
19:             double x, y;
20: 
21:             for(i = 0; i <= SIZE; ++i) {
22:                 for(j = 0; j <= SIZE; ++j) {
23:                     x = (double)(i - SIZE/2);
24:                     y = (double)(j - SIZE/2);
25:                     pos[i*(SIZE+1)+j] = new fk_Vector(x, y, (x*x - y*y)/(double)SIZE);
26:                 }
27:             }
28: 
29:             for(i = 0; i < SIZE; i++) {
30:                 for(j = 0; j < SIZE; j++) {
31:                     IFSet[(i*SIZE + j)*4 + 0] = i*(SIZE+1) + j;
32:                     IFSet[(i*SIZE + j)*4 + 1] = (i+1)*(SIZE+1) + j;
33:                     IFSet[(i*SIZE + j)*4 + 2] = (i+1)*(SIZE+1) + j+1;
34:                     IFSet[(i*SIZE + j)*4 + 3] = i*(SIZE+1) + j+1;
35:                 }
36:             }
37: 
38:             shape.MakeIFSet(SIZE*SIZE, 4, IFSet, (SIZE+1)*(SIZE+1), pos);
39:             viewer.Shape = shape;
40:             viewer.DrawMode = fk_Draw.FRONTBACK_FACE | fk_Draw.LINE;
41:             viewer.Scale = 10.0;
42: 
43:             for(int counter = 0; viewer.Draw() == true; counter += 10) {
44:                 for(i = 0; i <= SIZE; i++) {
45:                     for(j = 0; j <= SIZE; j++) {
46:                         moveVec.Set(0.0, 0.0, Math.Sin((double)(counter + j*40)*0.05/Math.PI));
47:                         movePos = moveVec + pos[i*(SIZE+1)+j];
48:                         shape.MoveVPosition(i*(SIZE+1)+j, movePos);
49:                     }
50:                 }
51:             }
52:         }
53:     }
54: }
\end{verbatim}
\end{small}
\end{breakbox}

\section{Boid アルゴリズムによる群集シミュレーション}
Boid アルゴリズムとは、群集シミュレーションを実現するための最も基本的なアルゴリズムである。
このアルゴリズムでは、群を構成する各エージェントは以下の3つの動作規則を持つ。

\begin{description}
\item[分離 (Separation):] ~ \\
	エージェントが、他のエージェントとぶつからないように距離を取る。
	\myfig{Fig:Boid-Sep}{./Fig/Boid-Sep.eps}{width=4truecm}{分離の概念図}{0mm}

\item[整列 (Alignment):] ~ \\
	エージェントが周囲のエージェントと方向ベクトルと速度ベクトルを合わせる。
	\myfig{Fig:Boid-Ali}{./Fig/Boid-Ali.eps}{width=4truecm}{整列の概念図}{0mm}

\item[結合 (Cohesion):] ~ \\
	エージェントは、群れの中心方向に集まるようにする。
	\myfig{Fig:Boid-Coh}{./Fig/Boid-Coh.eps}{width=4truecm}{結合の概念図}{0mm}

\end{description}
これを実現するための計算式は、以下の通りである。
なお、\(n\)個のエージェントの識別番号を\(i \; (0, 1, \cdots, n-1)\) とし、
それぞれの位置ベクトル、速度ベクトルを \(\bP_i, \bV_i\) とする。

\begin{description}
 \item[分離:] ~ \\
エージェント\(i\)が\(j\)から離れたければ、元の速度ベクトルに対し
\(\lvec{\bP_j\bP_i} = \bP_i - \bP_j\) を合わせれば良いので、
\(\alpha\)をやや小さめの適当な数値として
\begin{equation}
	{\bV_i}' = \bV_i + \alpha \frac{\bP_i - \bP_j}{|\bP_i - \bP_j|}
\end{equation}
とすればよい。これを近隣にあるエージェント全てにおいて計算すればよい。
近隣エージェントの番号集合を\(N\)とすれば、
\begin{equation}
	{\bV_i}' = \bV_i + \alpha \sum_{j \in N}\frac{\bP_i - \bP_j}{|\bP_i - \bP_j|}
\end{equation}
となる。

\item[整列:] ~ \\
周囲のエージェントの速度ベクトルの平均を出し、その分を自身の速度ベクトルに足せばよい。
数式としては以下の通り。
\begin{equation}
	{\bV_i}' = \frac{\bV_i + \beta\sum_{j \in N}\bV_j}{|\bV_i + \beta\sum_{j \in N}\bV_j|} .
\end{equation}

\item[結合:] ~ \\
群れ全体の重心\(\bG\)を算出し、そこに向かうようなベクトルを速度ベクトルに追加する。
\begin{equation}
	{\bV_i}' = \bV_i + \gamma (\frac{\sum\bP_i}{n} - \bP_i) .
\end{equation}
\end{description}

以下、C\# によるサンプルプログラムを掲載する。
これらついての詳細な解説は割愛するが、
各エージェントを表す「Agent」クラスと、群を表す「Boid」クラスを構築することにより、
メインループがかなりシンプルになっていることがわかる。また、
「S」「A」「C」キーを押すと、それぞれ「分離」「整列」「結合」規則を無効にするように
なっているので、各規則がどのようにエージェントの動作に影響しているかがわかるだろう。\\ ~
\begin{breakbox}
\begin{small}
\begin{verbatim}
  1: using System;
  2: using FK_CLI;
  3: 
  4: namespace FK_CLI_Boid
  5: {
  6:     // エージェント用クラス
  7:     class Agent
  8:     {
  9:         private fk_Model model;
 10:         private fk_Vector newVec;
 11:         private const double SPEED = 0.05;   // 速度設定値
 12:         public const double AREASIZE = 15.0; // 移動領域の広さ設定値
 13: 
 14:         // コンストラクタ
 15:         public Agent()
 16:         {
 17:             model = new fk_Model();
 18:             model.Material = fk_Material.Red;
 19:             model.GlVec(fk_Math.DRand(-1.0, 1.0), fk_Math.DRand(-1.0, 1.0), 0.0);
 20:             model.GlMoveTo(fk_Math.DRand(-AREASIZE, AREASIZE),
 21:                            fk_Math.DRand(-AREASIZE, AREASIZE), 0.0);
 22:         }
 23: 
 24:         // 位置ベクトル用プロパティ
 25:         public fk_Vector Pos
 26:         {
 27:             get
 28:             {
 29:                 return model.Position;
 30:             }
 31:         }
 32: 
 33:         // 方向ベクトル用プロパティ
 34:         public fk_Vector Vec
 35:         {
 36:             set
 37:             {
 38:                 newVec = value;
 39:             }
 40:             get
 41:             {
 42:                 return model.Vec;
 43:             }
 44:         }
 45: 
 46:         // 形状参照プロパティ
 47:         public fk_Shape Shape
 48:         {
 49:             set
 50:             {
 51:                 model.Shape = value;
 52:             }
 53:         }
 54: 
 55:         // ウィンドウへのモデル登録
 56:         public void Entry(fk_AppWindow argWin)
 57:         {
 58:             argWin.Entry(model);
 59:         }
 60: 
 61:         // 前進処理
 62:         public void Forward()
 63:         {
 64:             model.GlVec(newVec);
 65:             model.LoTranslate(0.0, 0.0, -SPEED);
 66:         }
 67:     }
 68: 
 69:     // 群衆用クラス
 70:     class Boid
 71:     {
 72:         private Agent [] agent; // エージェント用配列
 73:         private fk_Cone cone;   // 形状 (円錐)
 74: 
 75:         private double paramA, paramB, paramC, paramLA, paramLB;
 76: 
 77:         public Boid(int argNum)
 78:         {
 79:             fk_Material.InitDefault();
 80:             cone = new fk_Cone(16, 0.4, 1.0);
 81:             if(argNum < 0) return;
 82:             agent = new Agent[argNum];
 83: 
 84:             for(int i = 0; i < argNum; ++i)
 85:             {
 86:                 agent[i] = new Agent();
 87:                 agent[i].Shape = cone;
 88:             }
 89: 
 90:             paramA = 0.2;
 91:             paramB = 0.02;
 92:             paramC = 0.01;
 93:             paramLA = 3.0;
 94:             paramLB = 5.0;
 95:         }
 96: 
 97:         public void SetParam(double argA, double argB, double argC, double argLA, double argLB)
 98:         {
 99:             paramA = argA;
100:             paramB = argB;
101:             paramC = argC;
102:             paramLA = argLA;
103:             paramLB = argLB;
104:         }
105: 
106:         public void SetWindow(fk_AppWindow argWin)
107:         {
108:             foreach(Agent M in agent)
109:             {
110:                 M.Entry(argWin);
111:             }
112:         }
113: 
114:         // 群集の更新処理
115:         public void Update(bool argSMode, bool argAMode, bool argCMode)
116:         {
117:             var gVec = new fk_Vector();
118:             var diff = new fk_Vector();
119:             fk_Vector [] pArray = new fk_Vector[agent.Length];
120:             fk_Vector [] vArray = new fk_Vector[agent.Length];
121: 
122:             // 位置ベクトルと速度ベクトルのコピー
123:             for(int i = 0; i < agent.Length; i++)
124:             {
125:                 pArray[i] = agent[i].Pos;
126:                 vArray[i] = agent[i].Vec;
127:                 gVec += pArray[i];
128:             }
129: 
130:             // 群集重心の算出
131:             gVec /= (double)(agent.Length);
132: 
133:             for(int i = 0; i < agent.Length; i++)
134:             {
135:                 fk_Vector p = new fk_Vector(pArray[i]);
136:                 fk_Vector v = new fk_Vector(vArray[i]);
137:                 for(int j = 0; j < agent.Length; j++)
138:                 {
139:                     if(i == j) continue;
140:                     diff = p - pArray[j];
141:                     double dist = diff.Dist();
142: 
143:                     // 分離 (Separation) 処理
144:                     if (dist < paramLA && argSMode == true)
145:                     {
146:                         v += paramA * diff / (dist*dist);
147:                     }
148: 
149:                     // 整列 (Alignment) 処理
150:                     if(dist < paramLB && argAMode == true)
151:                     {
152:                         v += paramB * vArray[j];
153:                     }
154:                 }
155: 
156:                 // 結合 (Cohesion) 処理
157:                 if(argCMode == true)
158:                 {
159:                     v += paramC * (gVec - pArray[i]);
160:                 }
161: 
162:                 // 領域の外側に近づいたら方向修正
163:                 if (Math.Abs(p.x) > Agent.AREASIZE && p.x * v.x > 0.0 && Math.Abs(v.x) > 0.01)
164:                 {
165:                     v.x -= v.x * (Math.Abs(pArray[i].x) - Agent.AREASIZE)*0.2;
166:                 }
167: 
168:                 if(Math.Abs(p.y) > Agent.AREASIZE && p.y * v.y > 0.0 && Math.Abs(v.y) > 0.01)
169:                 {
170:                     v.y -= v.y * (Math.Abs(pArray[i].y) - Agent.AREASIZE)*0.2;
171:                 }
172: 
173:                 v.z = 0.0;
174:                 agent[i].Vec = v;
175:             }
176: 
177:             foreach(Agent M in agent) {
178:                 M.Forward();
179:             }
180:         }
181:     }
182: 
183:     class Program
184:     {
185:         static void Main(string[] args)
186:         {
187:             var win = new fk_AppWindow();
188:             var boid = new Boid(200);
189: 
190:             boid.SetWindow(win);
191: 
192:             // ウィンドウ各種設定
193:             win.Size = new fk_Dimension(800, 800);
194:             win.BGColor = new fk_Color(0.6, 0.7, 0.8);
195:             win.ShowGuide(fk_Guide.GRID_XY);
196:             win.CameraPos = new fk_Vector(0.0, 0.0, 80.0);
197:             win.CameraFocus = new fk_Vector(0.0, 0.0, 0.0);
198:             win.TrackBallMode = true;
199: 
200:             win.Open();
201: 
202:             while(win.Update() == true)
203:             {
204:                 // Sキーで「Separate(分離)」を無効に
205:                 bool sMode = win.GetKeyStatus('S', fk_Switch.RELEASE);
206: 
207:                 // Aキーで「Alignment(整列)」を無効に
208:                 bool aMode = win.GetKeyStatus('A', fk_Switch.RELEASE);
209: 
210:                 // Cキーで「Cohesion(結合)」を無効に
211:                 bool cMode = win.GetKeyStatus('C', fk_Switch.RELEASE);
212: 
213:                 // 群集の更新処理
214:                 boid.Update(sMode, aMode, cMode);
215:             }
216:         }
217:     }
218: }
\end{verbatim}
\end{small}
\end{breakbox}

\section{パーティクルアニメーション}
パーティクルアニメーションとは、粒子の移動によって気流や水流などを
表現する手法である。FK システムでは、パーティクルアニメーションを作成する
ためのクラスとして fk\_Particle 及び fk\_ParticleSet クラスを用意している。
これらの細かな仕様に関しては\ref{sec:particle} 節に記述してあるが、
ここではサンプルプログラムを用いておおまかな利用法を説明する。

fk\_ParticleSet クラスは、これまで紹介したクラスとはやや利用手法が
異なっている。まず、fk\_ParticleSet クラスを継承したクラスを作成し、
いくつかの仮想関数に対して再定義を行う。あとは、getShape() 関数を
利用して fk\_Model に形状として設定したり、fk\_ShapeViewer を利用して
描画することができる。

ここでは、サンプルとして円柱の周囲を流れる水流のシミュレーションの様子を
描画するプログラムを紹介する。
\begin{center}
詳細解説
\end{center}
\begin{itemize}
 \item 8〜75行目でパーティクルに対する各種設定を行っている。
	10行目でパーティクル用インスタンスを生成し、
	11〜13行目で各種設定を行っている。

 \item 11 行目の MaxSize プロパティは fk\_ParticleSet クラスのメンバで、
	パーティクル個数の最大値を設定する。もしパーティクルの個数が
	この値と等しくなったときは、NewParticle() メソッドを呼んでも
	パーティクルは新たに生成されなくなる。

 \item 12,13 行目はそれぞれ個別処理、全体処理に対するモード設定である。
	ここで true に設定しない場合、IndivMethod() や AllMethod() の
	記述は無視される。

 \item 16 〜 22 行目では、新たにパーティクルが生成された際の処理を記述する。
	引数の P に新パーティクルのインスタンスが入っており、これに対して
	様々な設定を行う。今回のサンプルでは初期位置設定を行っている。

	ここにある GenMethod() や AllMethod(), IndivMethod() は
	fk\_Particle 型を引数に持つ void 型のメソッドか、
	またはラムダ式で設定を行うことができる。
	今回のサンプルではラムダ式を用いている。
	ラムダ式についての詳細は C\# に関する一般的な書籍を参照されたい。

 \item 25 〜 34 行目では、AllMethod() メソッドを再定義している。
	AllMethod() メソッドには、パーティクル集合全体に対しての処理を記述する。
	ここではランダムにパーティクルの生成を行っているだけであるが、
	パーティクル全体に対して一括の処理を記述することもできる。

 \item 37 〜 71 行目では、IndivMethod() メソッドを再定義している。
	IndivMethod 関数には、個別のパーティクルに対する処理を記述する。

 \item IndivMethod() 中では、52 〜 55 行目で速度ベクトルの設定を行っている。
	中心が原点で、\(z\) 軸に平行な半径 \(R\) の円柱の周囲を
	速度 \((-V_x, 0, 0)\) の水流が流れているとする。このとき、各地点
	\((x, y, z)\) での水流を表す偏微分方程式は以下のようなものである。
	\begin{equation}
		\frac{\partial}{\partial t}\bP =
		\bV + \frac{R^3}{2}
		\left(\frac{\bV}{r^3} -
		\frac{3 \bV \cdot \bP}{r^5} \bP \right) .
	\end{equation}
	ただし、
	\begin{equation}
		\bV = (-V_x, 0, 0), \qquad
		\bP = (x, y, 0), \qquad r = |\bP|
	\end{equation}
	である。今回は、\(\bV = (0.2, 0, 0)\)(40行目の「water」変数)、
	\(R = 15\)(41行目の「R」変数) として算出している。この式から、
	各パーティクルの速度ベクトルを算出し、55 行目で設定している。

 \item パーティクルの色は、速度が minSpeed 未満の場合は青、
	maxSpeed 以上の場合は赤とし、その中間の場合は
	赤色と青色をブレンドした色となるように設定している。
	その色値の算出と設定を 58 〜 64 行目で行っている。
	パーティクルの速度 \(s\) が \(m < s < M\) を満たすとき、
	\begin{equation}
		t = \frac{s - m}{M - m}
	\end{equation}
	とし、赤色値を\(R\)、青色値を\(B\)としたとき
	\begin{equation}
		 C = (1-t)B + tR
	\end{equation}
	という式で色値\(C\)を決定している。

 \item 67 〜 70 行目でパーティクル削除判定を行っている。
	パーティクルが \(x = -50\) よりも左へ流れてしまった場合には
	69 行目で削除を行っている。

 \item パーティクル集合を表示するには、89 行目にあるように
	fk\_ParticleSet クラスの Shape プロパティを
	fk\_Model の Shape プロパティに代入する必要がある。

 \item 136 行目にあるように、Handle() メソッドを用いることで
	パーティクル全体に 1 ステップ処理が行われる。その際には、設定した
	速度や加速度にしたがって各パーティクルが移動する。特に再設定
	しない限り、加速度は処理終了後も保存される。
	今回のプログラムではパーティクルの加速度は一切設定していないため、
	IndivMethod() メソッド内の 55 行目の速度設定のみが
	パーティクルの動作を決定する要因となる。

\end{itemize}

\begin{breakbox}
\begin{small}
\begin{verbatim}
  1: using System;
  2: using FK_CLI;
  3: 
  4: namespace FK_CLI_Particle
  5: {
  6:     class Program
  7:     {
  8:         static fk_ParticleSet ParticleSetup()
  9:         {
 10:             var particle = new fk_ParticleSet();
 11:             particle.MaxSize = 1000; // パーティクル最大数
 12:             particle.AllMode = true; // AllMethod() の有効化
 13:             particle.IndivMode = true; // IndivMethod() の有効化
 14: 
 15:             // パーティクル生成時処理をラムダ式で設定
 16:             particle.GenMethod = (P) =>
 17:             {
 18:                 // 生成時の位置を(ランダムに)設定
 19:                 double y = fk_Math.DRand(-25.0, 25.0);
 20:                 double z = fk_Math.DRand(-25.0, 25.0);
 21:                 P.Position = new fk_Vector(50.0, y, z);
 22:             };
 23: 
 24:             // パーティクル全体処理をラムダ式で設定
 25:             particle.AllMethod = () =>
 26:             {
 27:                 for (int i = 0; i < 5; i++)
 28:                 {
 29:                     if (fk_Math.DRand() < 0.3)
 30:                     {   // 発生確率は 30% (を5回)
 31:                         particle.NewParticle();              // パーティクル生成処理
 32:                     }
 33:                 }
 34:             };
 35: 
 36:             // パーティクル個別処理をラムダ式で設定
 37:             particle.IndivMethod = (P) =>
 38:             {
 39:                 fk_Vector pos, vec, tmp1, tmp2;
 40:                 var water = new fk_Vector(-0.5, 0.0, 0.0);
 41:                 double R = 15.0;
 42:                 double minSpeed = 0.3;
 43:                 double maxSpeed = 0.6;
 44:                 double r;
 45: 
 46:                 //Console.WriteLine("count A {0}", P.ID);
 47:                 pos = P.Position;        // パーティクル位置取得。
 48:                 pos.z = 0.0;
 49:                 r = pos.Dist();          // |p| を r に代入。
 50: 
 51:                 // パーティクルの速度ベクトルを計算
 52:                 tmp1 = water / (r * r * r);
 53:                 tmp2 = ((3.0 * (water * pos)) / (r * r * r * r * r)) * pos;
 54:                 vec = water + ((R * R * R) / 2.0) * (tmp1 - tmp2);
 55:                 P.Velocity = vec;
 56: 
 57:                 // パーティクルの色を計算
 58:                 double speed = vec.Dist();
 59:                 double s = (speed - minSpeed) / (maxSpeed - minSpeed);
 60:                 double t = Math.Min(1.0, Math.Max(0.0, s));
 61:                 double h = Math.PI * (4.0 + 2.0 * t)/3.0;
 62:                 var col = new fk_Color();
 63:                 col.SetHSV(h, 1.0, 1.0);
 64:                 P.Color = col;
 65: 
 66:                 // パーティクルの x 成分が -50 以下になったら消去
 67:                 if (pos.x < -50.0)
 68:                 {
 69:                     particle.RemoveParticle(P);
 70:                 }
 71:             };
 72: 
 73:             // Main() にインスタンスを返す。
 74:             return particle;
 75:         }
 76: 
 77:         static fk_AppWindow WindowSetup()
 78:         {
 79:             var window = new fk_AppWindow();
 80:             window.Size = new fk_Dimension(800, 800);
 81:             window.BGColor = new fk_Color(0.0, 0.0, 0.0);
 82:             window.TrackBallMode = true;
 83:             return window;
 84:         }
 85: 
 86:         static void ParticleModelSetup(fk_ParticleSet argParticle, fk_AppWindow argWindow)
 87:         {
 88:             var model = new fk_Model();
 89:             model.Shape = argParticle.Shape;
 90:             model.DrawMode = fk_Draw.POINT;
 91: 
 92:             // パーティクルごとに色を変えたい場合の設定
 93:             model.ElementMode = fk_ElementMode.ELEMENT;
 94: 
 95:             // パーティクル描画の際の大きさ (ピクセル)
 96:             model.PointSize = 5.0;
 97: 
 98:             argWindow.Entry(model);
 99:         }
100: 
101:         static void PrismSetup(fk_AppWindow argWindow)
102:         {
103:             var prism = new fk_Prism(40, 15.0, 15.0, 50.0);
104:             var prismModel = new fk_Model();
105:             prismModel.Shape = prism;
106:             prismModel.GlMoveTo(0.0, 0.0, 25.0);
107:             prismModel.DrawMode = fk_Draw.FACE | fk_Draw.LINE | fk_Draw.POINT;
108:             prismModel.Material = fk_Material.Yellow;
109:             prismModel.LineColor = new fk_Color(0.0, 0.0, 1.0);
110:             prismModel.PointColor = new fk_Color(0.0, 1.0, 0.0);
111:             argWindow.Entry(prismModel);
112:         }
113: 
114:         static void Main(string[] args)
115:         {
116:             fk_Material.InitDefault();
117: 
118:             // パーティクルセットの生成
119:             var particle = ParticleSetup();
120: 
121:             // ウィンドウ設定
122:             var window = WindowSetup();
123: 
124:             // パーティクルモデル設定
125:             ParticleModelSetup(particle, window);
126: 
127:             // 内部円柱設定
128:             PrismSetup(window);
129: 
130:             // ウィンドウ生成
131:             window.Open();
132: 
133:             // メインループ
134:             while (window.Update())
135:             {
136:                 particle.Handle(); // パーティクルを 1 ステップ実行する。
137:             }
138:         }
139:     }
140: }
\end{verbatim}
\end{small}
\end{breakbox}

\section{音再生}

FK では音再生用のクラスとして幾つかのクラスが用意されているが、
そのうちユーザーが簡易に用いることを想定したクラスは fk\_Sound, fk\_BGM の2つである。

fk\_BGM クラスは音楽 (Back Ground Music) を再生するためのクラスで、
基本的な利用はまずコンストラクタで BGM ファイルを指定し、
その後 Start() で再生開始、Gain プロパティで音量制御というシンプルなものとなっている。

fk\_Sound は効果音(Sound Effect, 以下「SE」)を再生するためのものである。
SE は BGM とは異なり任意のタイミングで再生が開始となること、
複数の SE が同時に発生することがあること、リピート再生が行われないことなどの
差異があるため、BGM とは別クラスとなっている。
このクラスの機能としては、コンストラクタで入力する SE のファイル数、
LoadData() で SE ファイルの指定、StartSE() で各 SE の再生開始というものとなっている。

両クラスとも、使用終了時やプログラムの終了時には「StopStatus」プロパティに
true を代入する必要がある。これを行わないと、プログラムが正常に終了しない可能性があるので
注意すること。
\\ ~

\begin{breakbox}
\begin{small}
\begin{verbatim}
  1: using System;
  2: using FK_CLI;
  3: 
  4: namespace FK_CLI_Audio
  5: {
  6:     class Program
  7:     {
  8:         static void Main(string[] args)
  9:         {
 10:             // マテリアル初期化
 11:             fk_Material.InitDefault();
 12: 
 13:             // ウィンドウの各種設定
 14:             var win = new fk_AppWindow();
 15:             WindowSetup(win);
 16: 
 17:             // 立方体の各種設定
 18:             var blockModel = new fk_Model();
 19:             BlockSetup(blockModel, win);
 20: 
 21:             // BGM用変数の作成
 22:             var bgm = new fk_BGM("epoq.ogg");
 23:             double volume = 0.5;
 24: 
 25:             // SE用変数の作成
 26:             var se = new fk_Sound(2);
 27:             // 音源読み込み (IDは0番)
 28:             se.LoadData(0, "MIDTOM2.wav");
 29:             // 音源読み込み (IDは1番)
 30:             se.LoadData(1, "SDCRKRM.wav");
 31: 
 32:             // ウィンドウ表示
 33:             win.Open();
 34: 
 35:             // BGM再生スタート
 36:             bgm.Start();
 37: 
 38:             // SE出力スタンバイ
 39:             se.Start();
 40: 
 41:             var origin = new fk_Vector(0.0, 0.0, 0.0);
 42:             while (win.Update())
 43:             {
 44:                 blockModel.GlRotateWithVec(origin, fk_Axis.Y, Math.PI / 360.0);
 45: 
 46:                 // volume値の変更
 47:                 volume = ChangeVolume(volume, win);
 48:                 bgm.Gain = volume;
 49: 
 50:                 // SE再生
 51:                 PlaySE(se, win);
 52:             }
 53: 
 54:             // BGM変数とSE変数に終了を指示
 55:             // (最後にこれをやらないといつまでも再生が止まらずプログラムが終了しません)
 56:             bgm.StopStatus = true;
 57:             se.StopStatus = true;
 58:         }
 59: 
 60:         // ウィンドウ設定メソッド
 61:         static void WindowSetup(fk_AppWindow argWin)
 62:         {
 63:             argWin.CameraPos = new fk_Vector(0.0, 1.0, 20.0);
 64:             argWin.CameraFocus = new fk_Vector(0.0, 1.0, 0.0);
 65:             argWin.Size = new fk_Dimension(600, 600);
 66:             argWin.BGColor = new fk_Color(0.6, 0.7, 0.8);
 67:             argWin.ShowGuide(fk_Guide.GRID_XZ);
 68:             argWin.TrackBallMode = true;
 69:             argWin.FPS = 60;
 70:         }
 71: 
 72:         // 立方体モデル設定メソッド
 73:         static void BlockSetup(fk_Model argModel, fk_AppWindow argWin)
 74:         {
 75:             // 立方体の各種設定
 76:             var block = new fk_Block(1.0, 1.0, 1.0);
 77:             argModel.Shape = block;
 78:             argModel.GlMoveTo(3.0, 3.0, 0.0);
 79:             argModel.Material = fk_Material.Yellow;
 80:             argWin.Entry(argModel);
 81:         }
 82: 
 83:         // 音量調整メソッド
 84:         static double ChangeVolume(double argVol, fk_AppWindow argWin)
 85:         {
 86:             double tmpV = argVol;
 87: 
 88:             // 上矢印キーで BGM 音量アップ
 89:             if (argWin.GetSpecialKeyStatus(fk_Key.UP, fk_Switch.DOWN) == true)
 90:             {
 91:                 tmpV += 0.1;
 92:                 if (tmpV > 1.0)
 93:                 {
 94:                     tmpV = 1.0;
 95:                 }
 96:             }
 97: 
 98:             // 下矢印キーで BGM 音量ダウン
 99:             if (argWin.GetSpecialKeyStatus(fk_Key.DOWN, fk_Switch.DOWN) == true)
100:             {
101:                 tmpV -= 0.1;
102:                 if (tmpV < 0.0)
103:                 {
104:                     tmpV = 0.0;
105:                 }
106:             }
107:             return tmpV;
108:         }
109: 
110:         // SE再生(トリガー)メソッド
111:         static void PlaySE(fk_Sound argSE, fk_AppWindow argWin)
112:         {
113:             // Z キーで 0 番の SE を再生開始
114:             if (argWin.GetKeyStatus('Z', fk_Switch.DOWN) == true)
115:             {
116:                 argSE.StartSound(0);
117:             }
118: 
119:             // X キーで 1 番の SE を再生開始
120:             if (argWin.GetKeyStatus('X', fk_Switch.DOWN) == true)
121:             {
122:                 argSE.StartSound(1);
123:             }
124:         }
125:     }
126: }
\end{verbatim}
\end{small}
\end{breakbox}

\section{スプライト表示}
この節では、第\ref{sec:spritemodel}節で紹介したスプライトモデルを用いた
文字列表示のサンプルを示す。ここではフォントファイルとして
「rm1b.ttf」というファイル名のフォントデータを利用している。
これはビルド時に生成される exe ファイルと同一の場所に置かれている必要がある。
フォントを設定するには 23 行目にあるように InitFont() メソッドを用いる。

31, 34 行目で Text プロパティに様々な設定を行っているが、
この Text プロパティは fk\_TextImage 型であり、fk\_TextImage が持つ様々な設定を
変更することで色やフォントサイズなど、様々な設定を変更することができる。
詳細はリファレンスマニュアルのクラスの説明を参照のこと。

28 行目の SetPositionLT() メソッドは表示位置を指定するものである。
この SetPositionLT() は 64 行目でもメインループの度に位置の再設定を行っているが、
その理由は文字列を表すテクスチャ画像の幅が変更されたときに再設定を行わないと、
表示位置がずれてしまうためである。
表示文字列が変更されていない場合や、
文字列テクスチャの幅が変わっていないことが保証されている場合は、
メインループ中にこのメソッドを呼ぶ必要はない。
実際、今回のサンプルでは(等幅設定を行い、描画文字数が変化しないことから)
文字盤の画像幅は終始変動しないため、
64 行目をコメントアウトしても正常に動作するが、
等幅設定を用いない場合などは文字列の変更の際に幅変動も伴うので、
このメソッドを呼ばないと表示が適切になされない場合がある。

\begin{breakbox}
\begin{small}
\begin{verbatim}
 1: using System;
 2: using FK_CLI;
 3: 
 4: namespace FK_CLI_Sprite
 5: {
 6:     class Program
 7:     {
 8:         static void Main(string[] args)
 9:         {
10:             var window = new fk_AppWindow();
11:             fk_Material.InitDefault();
12:             
13:             // 文字盤用変数の生成
14:             var sprite = new fk_SpriteModel();
15: 
16:             var block = new fk_Block(1.0, 1.0, 1.0);
17:             var model = new fk_Model();
18:             var origin = new fk_Vector(0.0, 0.0, 0.0);
19:             int count;
20:             string str, space;
21:  
22:             // 文字盤に対するフォントの読み込み
23:             if(sprite.InitFont("rm1b.ttf") == false) {
24:                 Console.WriteLine("Font Error");
25:             }
26: 
27:             // 文字盤の表示位置設定
28:             sprite.SetPositionLT(-330.0, 240.0);
29: 
30:             // 文字盤のフォントを等幅に設定
31:             sprite.Text.MonospaceMode = true;
32: 
33:             // 文字盤の(等幅)サイズを設定
34:             sprite.Text.MonospaceSize = 12;
35:             window.Entry(sprite);
36:  
37:             model.Shape = block;
38:             model.GlMoveTo(0.0, 6.0, 0.0);
39:             model.Material = fk_Material.Yellow;
40:             window.Entry(model);
41:             window.CameraPos = new fk_Vector(0.0, 5.0, 20.0);
42:             window.CameraFocus = new fk_Vector(0.0, 5.0, 0.0);
43:             window.Size = new fk_Dimension(800, 600);
44:             window.BGColor = new fk_Color(0.6, 0.7, 0.8);
45:             window.TrackBallMode = true;
46:             window.Open();
47:             window.ShowGuide(fk_Guide.GRID_XZ);
48:  
49:             count = 0;
50:             while(window.Update() == true) {
51:                 // 数値の桁数によりスペースの個数を調整
52:                 if (count < 10) space = "   ";
53:                 else if(count < 100) space = "  ";
54:                 else if(count < 1000) space = " ";
55:                 else space = "";
56: 
57:                 // 「count = 数値」の文字列を生成
58:                 str = "count = " + space + count.ToString();
59: 
60:                 // 文字列を文字盤に入力
61:                 sprite.DrawText(str, true);
62: 
63:                 // 文字盤表示位置の再設定
64:                 sprite.SetPositionLT(-330.0, 240.0);
65: 
66:                 model.GlRotateWithVec(origin, fk_Axis.Y, Math.PI/360.0);
67:                 count++;
68:             }
69:         }
70:     }
71: }
\end{verbatim}
\end{small}
\end{breakbox}

\section{四元数}
この節では、第\ref{sec:quaternion}節で紹介した四元数(クォータニオン)を用いて、
姿勢補間を行うサンプルプログラムを示す。
3DCGのプログラミングでは、ベクトル、行列、オイラー角、四元数といった多くの
代数要素を扱う必要があるが、このサンプルプログラムはそれらの利用方法を
コンパクトにまとめたものとなっている。

四元数は姿勢を表現する手段として強力な数学手法であるが、
四元数の成分を直接扱うことは現実的ではなく、
通常はオイラー角を介して制御を行う。
このサンプルプログラムでも、まず2種類の姿勢を angle1, angle2 という変数で
設定(18,19行目)してから、それを四元数に変換(45,46行目)している。
そして、52行目で球面線形補間を行った四元数を算出し、
それを55行目でモデルの姿勢として設定している。\\ ~

\begin{breakbox}
\begin{small}
\begin{verbatim}
 1: using System;
 2: using FK_CLI;
 3: 
 4: namespace FK_CLI_Quaternion
 5: {
 6:     class Program
 7:     {
 8:         static void Main(string[] args)
 9:         {
10:             var win = new fk_AppWindow();
11:             win.Size = new fk_Dimension(800, 800);
12:             var model = new fk_Model();
13:             var pointM = new fk_Model();
14:             var cone = new fk_Cone(3, 4.0, 15.0);
15:             var pos = new fk_Vector(0.0, 0.0, -15.0);
16: 
17:             // オイラー角用変数の生成
18:             var angle1 = new fk_Angle(0.0, 0.0, 0.0);
19:             var angle2 = new fk_Angle(Math.PI/2.0, Math.PI/2.0 - 0.01, 0.0);
20: 
21:             // 四元数用変数の生成
22:             var q1 = new fk_Quaternion();
23:             var q2 = new fk_Quaternion();
24:             fk_Quaternion q;
25: 
26:             // 角錐頂点の軌跡用
27:             var point = new fk_Polyline();
28: 
29:             fk_Material.InitDefault();
30: 
31:             model.Shape = cone;
32:             model.Material = fk_Material.Yellow;
33:             model.GlAngle(angle1); // モデルのオイラー角を (0, 0, 0) に
34: 
35:             pointM.Shape = point;
36:             pointM.LineColor = new fk_Color(1.0, 0.0, 0.0);
37: 
38:             win.BGColor = new fk_Color(0.7, 0.8, 0.9);
39:             win.Entry(model);
40:             win.Entry(pointM);
41:             win.TrackBallMode = true;
42:             win.ShowGuide();
43:             win.Open();
44: 
45:             q1.Euler = angle1; // q1 にオイラー角 (0, 0, 0) を意味する四元数を設定
46:             q2.Euler = angle2; // q2 にオイラー角 (π/2, π/2-0.01, 0) を意味する四元数を設定
47: 
48:             double t = 0.0;
49: 
50:             while(win.Update() == true) {
51:                 // q に q1 と q2 を球面補間した値を設定
52:                 q = fk_Math.QuatInterSphere(q1, q2, t);
53: 
54:                 // モデルの姿勢を q に設定
55:                 model.GlAngle(q.Euler);
56: 
57:                 // 頂点軌跡の追加
58:                 if(t < 1.0) {
59:                     point.PushVertex(model.Matrix * pos);
60:                     t += 0.005;
61:                 }
62:             }
63:         }
64:     }
65: }
\end{verbatim}
\end{small}
\end{breakbox}
