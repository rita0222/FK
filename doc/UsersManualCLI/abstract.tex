\chapter*{はじめに}
\addcontentsline{toc}{chapter}{\protect\numberline{}
					{はじめに}}
~

この文章で述べられている FK (Fine Kernel)  System は、
容易にインタラクティブな 3D 空間を表現するための Tool Kit である。

ここでいう Tool Kit とは、システムを構築する際に
用いられる簡易インターフェースをプログラミング言語から呼び出す
形で実現されたもののことをいう。平易な言葉で述べるなら、この
FK System を用いれば簡単にインタラクティブな 3D の世界を創造することが
可能であるということである。普通、なんらかのシステム構築の際には
本質的でない部分に労力をさかねばならないことは周知のとおりである。
それは、ときには学習であったり、ときには作業であったり、ときには
試行錯誤であったりする。ツールキットは、それらをユーザに代わって
肩代りをし、より本質的な部分にのみユーザが没頭することを助ける
役割を持つ。

FK System が、3D 空間の作成をサポートすることは前述したが、
大きな理念としての柱が幾つかある。それを列挙すると、
\begin{itemize}
 \item オブジェクト指向概念の採用。
 \item モデルに対する制御の柔軟性。
 \item 形状の容易な定義や変形。
 \item 複雑な座標系処理の簡便化。
 \item ディスプレイリストの概念。
 \item インターフェースの柔軟な構築。
 \item 汎用性と高速描画の両立。
 \item 環境との非依存。
\end{itemize}
といった事柄を特に重要視して設計が行われている。

これらの概念を、我々はまず C++ 言語を用いたクラスライブラリとして実現した。
さらに Ver.3 では、この C++ 版をベースとして CLI による実装も追加した。
CLI 版を用いることで、C++ 版とほぼ同一の機能を C\# や F\# といった
.NET 対応言語で開発することが可能となった。
C++ と C\# の両方に共通な 3D フレームワークはあまり多くはなく、
両方の言語をシームレスに扱うことができるという点が、FK System のユニークな点である。
また、F\# のような関数型言語においては 3D プログラミングフレームワークはあまり普及しておらず、
関数型プログラミングを用いた 3D プログラミングを行いたい場合で、
FK は大変有用なライブラリとなるであろう。

本書は、\ref{chap:sample} 章で構成されている。

第 \ref{chap:intro} 章では、FK System の基本的な考え方を理解するため、
簡単なサンプルを用いて機能を紹介していく。

第 \ref{chap:vector} 章では、FK システムで準備された三次元座標値や三次元ベクトル
に関しての扱い方を述べる。座標やベクトルは、特に第 \ref{chap:shape}、
%\ref{chap:solid}、
\ref{chap:model} 章の
内容と著しく関わる。形状は、もちろん三次元座標で表現されるし、
モデルの挙動の制御にはベクトルや座標を多用するからである。

第 \ref{chap:material} 章では、マテリアルと呼ばれるカラー属性に関して述べる。
これは、形状や光源に対して色を含む質感を設定する際に
用いられるパラメータのことである。非常に細かな設定が可能であるが、
簡易な使用法もあることをここでは述べている。

第 \ref{chap:shape} 章では、
形状の扱いに関して述べる。\ref{chap:shape} 章では、
基本形状の生成法を中心に述べる。さらに、
\ref{chap:easygen} 章では任意形状の動的な生成や変形方法に関して
解説する。

% \ref{chap:solid} 章では、かなり高度な
% 形状の生成、変形、参照方法に関してを解説する。
% \ref{chap:solid} 章の内容は高度な知識を要求するため、
% 変形や位相参照の必要がないのであれば読み飛ばしても
% 差し支えない。

第 \ref{chap:imagetexture} 章では、FK の中で画像表示およびテクスチャマッピングを
行うための方法について述べる。さらに、特別なテクスチャマッピングとして「文字列テクスチャ」を
第 \ref{chap:stringimage} 章で解説する。画面上に文字を表示したい場合には、
この2つの章を参考にしてほしい。

第 \ref{chap:model}、\ref{chap:scene}、\ref{chap:window} 章は、
FK に関する大きな 3 つの概念 --- すなわち、
オブジェクト、シーン、ウィンドウ --- に対しての詳細な説明
を与えている。
これらと形状を含めた 4 つの関係は単純明解な包有関係で、
形状はオブジェクトに、オブジェクトはシーンに、
ディスプレイリストはウィンドウに設定される。

第 \ref{chap:viewer} 章では、簡易な形状描画手段に関しての解説を
述べる。

最終章は、全体を通しての様々なトピックやエッセンス、そして簡単な
例題を掲載する。
