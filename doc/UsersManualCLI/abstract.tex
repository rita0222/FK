\chapter*{はじめに}
\addcontentsline{toc}{chapter}{\protect\numberline{}
					{はじめに}}
~

この文章で述べられている FK (Fine Kernel)  System は、
容易にインタラクティブな 3D 空間を表現するための Tool Kit である。

ここでいう Tool Kit とは、システムを構築する際に
用いられる簡易インターフェースをプログラミング言語から呼び出す
形で実現されたもののことをいう。平易な言葉で述べるなら、この
FK System を用いれば簡単にインタラクティブな 3D の世界を創造することが
可能であるということである。普通、なんらかのシステム構築の際には
本質的でない部分に労力をさかねばならないことは周知のとおりである。
それは、ときには学習であったり、ときには作業であったり、ときには
試行錯誤であったりする。ツールキットは、それらをユーザに代わって
肩代りをし、より本質的な部分にのみユーザが没頭することを助ける
役割を持つ。

FK System が、3D 空間の作成をサポートすることは前述したが、
大きな理念としての柱が幾つかある。それを列挙すると、
\begin{itemize}
 \item オブジェクト指向概念の採用。
 \item モデルに対する制御の柔軟性。
 \item 形状の容易な定義や変形。
 \item 複雑な座標系処理の簡便化。
 \item ディスプレイリストの概念。
 \item インターフェースの柔軟な構築。
 \item 汎用性と高速描画の両立。
 \item 環境との非依存。
\end{itemize}
といった事柄を特に重要視して設計が行われている。これらは、
C++ 言語を用いたクラスライブラリという形で実現された。しかし、
C++ 言語に関しての知識がそれほどなくても、C 言語に対して
行ってきた開発プロセスをそのまま用いることができる。逆に、
C 言語のユーザはこのシステムを用いることによって C++ 言語の
クラスライブラリの有用性に対して驚くことだろう。

C++ では、型の概念に従来の C 言語のようなデータ構造のみでなく、
データを制御するための関数群も付属する。オブジェクト指向の
概念においては、これらのことを「メソッド」とか「メッセージ」
などと呼ぶこともある。この理念に基づくと、クラスライブラリの
ユーザはオブジェクトをメッセージによって操ることによって処理を
実現することとなる。実際、FK System はそれを念頭において開発が進められた。

本書は、\ref{sec:sample} 章で構成されている。

第 \ref{sec:intro} 章では、FK System の基本的な考え方を理解するため、
簡単なサンプルを用いて機能を紹介していく。

第 \ref{sec:vector} 章では、FK システムで準備された三次元座標値や三次元ベクトル
に関しての扱い方を述べる。座標やベクトルは、特に第 \ref{sec:shape}、
\ref{sec:reference}、\ref{sec:model} 章の
内容と著しく関わる。形状は、もちろん三次元座標で表現されるし、
モデルの挙動の制御にはベクトルや座標を多用するからである。

第 \ref{sec:material} 章では、マテリアルと呼ばれるカラー属性に関して述べる。
これは、形状や光源に対して色を含む質感を設定する際に
用いられるパラメータのことである。非常に細かな設定が可能であるが、
簡易な使用法もあることをここでは述べている。

第 \ref{sec:shape} 〜 \ref{sec:reference} 章では
形状の扱いに関して述べる。\ref{sec:shape} 章では、
基本形状の生成法を中心に述べる。さらに、
\ref{sec:easygen} 章では任意形状の動的な生成や変形方法に関して
解説する。\ref{sec:reference} 章では、かなり高度な
形状の生成、変形、参照方法に関してを解説する。
\ref{sec:reference} 章の内容は高度な知識を要求するため、
変形や位相参照の必要がないのであれば読み飛ばしても
差し支えない。

第 \ref{sec:model}、\ref{sec:scene}、\ref{sec:window} 章は、
FK に関する大きな 3 つの概念 --- すなわち、
オブジェクト、シーン、ウィンドウ --- に対しての詳細な説明
を与えている。
これらと形状を含めた 4 つの関係は単純明解な包有関係で、
形状はオブジェクトに、オブジェクトはシーンに、
ディスプレイリストはウィンドウに設定される。

第 \ref{sec:viewer} 章では、簡易な形状描画手段に関しての解説を
述べる。

最終章は、全体を通しての様々なトピックやエッセンス、そして簡単な
例題を掲載する。
