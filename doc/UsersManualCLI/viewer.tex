\chapter{簡易形状表示システム} \label{chap:viewer} ~

\ref{chap:window} 章では、FK システムでの多彩なウィンドウやデバイス制御機能を
紹介したが、中には FK システムで作成した形状を簡単に表示し、
様々な角度から閲覧したいという用途に用いたい利用者もいるであろう。
そのようなユーザは、fk\_AppWindow や fk\_Window によって
閲覧する様々な機能を構築するのはやや手間である。
そこで、FK システムではより簡単に形状を表示する手段として
fk\_ShapeViewer というクラスを提供している。

\section{形状表示}
利用するには、まず fk\_ShapeViewer 型の変数を定義する。\\ ~ \\
\begin{screen}
\begin{verbatim}
        var viewer = new fk_ShapeViewer(600, 600);
\end{verbatim}
\end{screen}
\\ ~

この時点で、多くの GUI が付加したウィンドウが生成される。形状は、
\ref{chap:shape} 章で紹介したいずれの種類でも利用できるが、ここでは例として
fk\_Solid 型及び fk\_Sphere 型の変数を準備する。
この変数が表す形状を表示するには、
SetShape() メソッドを用いる。SetShape() メソッドは二つの引数を
取り、最初の引数は立体 ID を表す整数、後ろの引数には形状を表す
変数(のアドレス)を入力する。複数の形状を一度に表示する場合は、
ID を変えて入力していくことで実現できる。\\ ~ \\
\begin{breakbox}
\begin{verbatim}
        var viewer = new fk_ShapeViewer(600, 600);
        var solid = new fk_Solid();
        var sphere = new fk_Sphere();

        viewer.SetShape(0, solid);
        viewer.SetShape(1, sphere);
\end{verbatim}
\end{breakbox}
~ \\

あとは、Draw() メソッドを呼ぶことで形状が描画される。通常は、次のように
繰り返し描画を行うことになる。\\ ~ \\
\begin{breakbox}
\begin{verbatim}
        while(viewer.Draw() == true) {
                // もし形状を変形するならば、ここに処理内容を記述する。
        }
\end{verbatim}
\end{breakbox}
~ \\

もし終了処理 (ウィンドウが閉じられる、「Quit」がメニューで選択されるなど)
が行われた場合、Draw() メソッドは false を返すので、その時点で while ループを
抜けることになる。形状変形の様子をアニメーション処理したい場合には、
while 文の中に変形処理を記述すればよい。具体的な変形処理のやり方は、
\ref{sec:movevertex} 節及び \ref{sec:sampleviewer} 節に
解説が記述されている。

\section{標準機能}
この fk\_ShapeViewer クラスで生成した形状ブラウザは、次のような機能を
GUI によって制御できる。これらの機能は、何もプログラムを記述することなく
利用することができる。

\begin{itemize}
 \item VRML、STL、DXF 各フォーマットファイル入力機能と、
	VRML ファイル出力機能。
 \item 表示されている画像をファイルに保存。
 \item 面画、線画、点画の各 ON/OFF 及び座標軸描画の ON/OFF。
 \item 光源回転有無の制御。
 \item 面画のマテリアル及び線画、点画での表示色設定。
 \item GUI によるヘディング、ピッチ、ロール角制御及び表示倍率、座標軸サイズの
	制御。
 \item 右左矢印キーによるヘディング角制御。
 \item スペースキーを押すことで表示倍率拡大。また、シフトキーを押しながら
	スペースキーを押すことで表示倍率縮小。
 \item マウスのドラッギングによる形状の平行移動。
\end{itemize}
