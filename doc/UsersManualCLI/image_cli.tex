\chapter{テクスチャマッピングと画像処理} \label{chap:imagetexture} ~

この章では、画像を 3D 空間上に表示する「テクスチャマッピング」と呼ばれる
技術の利用方法と、画像処理機能の使用方法を述べる。
基本的に、テクスチャマッピングは \ref{chap:shape} 章で述べてきた
形状の一種であり、利用方法は他の形状クラスとあまりかわらない。
しかし、画像情報を扱うため独特の機能を多く保持するため、独立した章で解説を行う。

\section{テクスチャマッピング} \label{sec:texture}
テクスチャマッピングとは、2次元画像の全部及び一部を3次元空間上に
配置して表示する技術である。テクスチャマッピングは、細かな質感を
簡単に表現できることや、高速な表示機能が搭載されているハードウェアが
普及してきていることから、非常に有用な技術である。FK システムでは、
現在「矩形テクスチャ」、「三角形テクスチャ」、「IFSテクスチャ」の
3種類のテクスチャマッピング方法をサポートしている。
現在、入力可能な画像フォーマットは Windows Bitmap 形式、PNG 形式、
JPEG 形式の3種類である。

\subsection{矩形テクスチャ}
最初に紹介するのは「矩形テクスチャ」と呼ばれるものである。これは、
2次元画像全体をそのまま(つまり長方形の状態で)表示するための機能で、
非常に簡単に利用できる。クラスとしては、「fk\_RectTexture」というものを
利用することになる。

\subsubsection{基本的な利用方法}
生成は、他の形状オブジェクトと同様に普通に変数を定義するだけでよい。
\\
\begin{screen}
\begin{verbatim}
        var texture = new fk_RectTexture();
\end{verbatim}
\end{screen}
~ \\
画像ファイルの入力は、Windows Bitmap 形式の場合 ReadBMP() メソッドを用いる。
このメソッドは、入力に成功した場合は true を、入力に失敗した場合は false を返す。
\\
\begin{breakbox}
\begin{verbatim}
        if(texture.ReadBMP("samp.bmp") == false)
        {
            Console.WriteLine("File Read Error");
        }
\end{verbatim}
\end{breakbox}
~ \\
PNG 形式や JPEG 形式の画像ファイルを読み込みたい場合は、上記の ReadBMP() を
ReadPNG(), ReadJPG() メソッドに置き換える。

また、テクスチャの3次元空間上での大きさの指定には TextureSize プロパティを用いる。
\\
\begin{screen}
\begin{verbatim}
        texture.TextureSize = new fk_TexCoord(50.0, 30.0);
\end{verbatim}
\end{screen}
~ \\
この状態で、中心を原点、向きを \(+z\) 方向としたテクスチャが生成される。

\subsubsection{画像中の一部分の切り出し}
fk\_RectTexture クラスでは、画像の一部分を切り出して表示することも可能である。
切り出し部分の指定方法として、「テクスチャ座標系」と呼ばれる座標系を用いる。
テクスチャ座標系というのは、
画像ファイルのうち一番左下の部分を \((0, 0)\)、右上の部分を \((1, 1)\) と
して、画像の任意の位置をパラメータとして表す座標系のことである。
例えば、画像の中心を表わすテクスチャ座標は \((0.5, 0.5)\) となる。
また、\(100 \times 100\) の画像の
左から 70 ピクセル、下から 40 ピクセルの位置のテクスチャ座標は
\((0.7, 0.4)\) ということになる。

切り出す部分は、切り出し部分の左下と右上のテクスチャ座標を設定することになる。
指定には SetTextureCoord() メソッドを利用する。以下は、
左下のテクスチャ座標として \((0.2, 0.3)\)、右上のテクスチャ座標として
\((0.5, 0.6)\) を指定するサンプルである。
\\
\begin{screen}
\begin{verbatim}
        texture.SetTextureCoord(0.2, 0.3, 0.5, 0.6);
\end{verbatim}
\end{screen}

\subsubsection{リピートモード}
ビルの外壁や地面を表すテクスチャを生成するときに、1枚の画像を
タイルのように行列状に並べて配置したい場合がある。これを全て別々の
テクスチャとして生成するのはかなり処理時間に負担がかかってしまう。
このようなとき、fk\_RectTexture の「リピートモード」を用いると便利である。
リピートモードとは、テクスチャを1枚だけ張るのではなく、タイル状に
並べて配置するモードである。これを用いるには、次のようにすればよい。
\\
\begin{screen}
\begin{verbatim}
        texture.TextureSize = new fk_TexCoord(100.0, 100.0);
        texture.RepeatMode = true;
        texture.RepeatParam = new fk_TexCoord(5.0, 10.0);
\end{verbatim}
\end{screen}
~ \\
RepeatMode プロパティはリピートモードを用いるかどうかを設定するもので、
true を代入するとリピートモードとなる。次の RepeatParam プロパティは
並べる個数を設定するもので、例の場合は横方向に 5 枚、縦方向に 10 枚の合計
50 枚を並べることになる。それら全体のサイズが 100x100 なので、1枚のサイズは
20x10 ということになる。ただし、リピートモードを用いる場合には画像サイズに
制限があり、縦幅と横幅はいずれも \(2^n\) (\(n\) は整数) である必要があり、
現在のサポートは \(2^{6} = 64\) から \(2^{16} = 65536\) までの間の
いずれかのピクセル幅でなければならない。
(ただし、縦幅と横幅は一致する必要はない。)
従って、リピートモードを用いるときはあらかじめ画像ファイルを補正しておく
必要がある。

また、リピートモードを用いた場合は一部の切り出しに関する設定は無効となる。

\subsection{三角形テクスチャ}
次に紹介するのは「三角形テクスチャ」である。これは、入力した画像の
一部分を三角形に切り出して表示する機能を持つ。これは、
「fk\_TriTexture」というクラスを利用する。テクスチャ用変数の
定義や画像ファイル読み込みに関しては fk\_RectTexture と同様である。
\\
\begin{screen}
\begin{verbatim}
        var texture = new fk_TriTexture();
        texture.ReadBMP("samp.bmp");
\end{verbatim}
\end{screen}
~ \\
fk\_TriTexture の場合も、fk\_RectTexture と同様に ReadBMP() を
ReadPNG() や ReadJPG() に置き換えることで、PNG 形式や JPEG 形式の画像ファイルを入力できる。

次に、画像のどの部分を切り出すかを指定する。
指定の方法は、前節で述べた「テクスチャ座標系」を
利用する。切り出す部分は、このテクスチャ座標系を利用して
3点それぞれを指定することになる。
指定には SetTextureCoord() メソッドを利用する。
最初の引数は、各頂点の ID を表わし、0, 1, 2 の順番で反時計回りとなるように
設定する。2番目、3番目の引数はテクスチャ座標の \(x, y\) 座標を入力する。
\\
\begin{screen}
\begin{verbatim}
        texture.SetTextureCoord(0, 0.0, 0.0);
        texture.SetTextureCoord(1, 1.0, 0.0);
        texture.SetTextureCoord(2, 0.5, 0.5);
\end{verbatim}
\end{screen}
~ \\
次に、3点の3次元空間上での座標を設定する。設定には setVertexPos() メソッドを
利用する。最初の引数が頂点 ID、2,3,4 番目の引数で 3 次元座標を指定する。
\\
\begin{screen}
\begin{verbatim}
        texture.SetVertexPos(0, 0.0, 0.0, 0.0);
        texture.SetVertexPos(1, 50.0, 0.0, 0.0);
        texture.SetVertexPos(2, 20.0, 30.0, 0.0);
\end{verbatim}
\end{screen}
~ \\
SetVertexPos() メソッドは、fk\_Vector 型の変数を引数に持たせることも可能である。
\\
\begin{screen}
\begin{verbatim}
        var vec = new fk_Vector(100.0, 0.0, 0.0);
        texture.SetVertexPos(0, vec);
\end{verbatim}
\end{screen}

\subsection{IFSテクスチャ} \label{subsec:ifstexture}
次に紹介する「IFSテクスチャ」は、多数の三角形テクスチャをひとまとめに
扱うための機能を持つクラスで、クラス名は「fk\_IFSTexure」である。
このクラスでは、Metasequoia によって作成したテクスチャ付きの MQO ファイルと、
D3DX ファイルの 2 種類のデータからの入力が可能となっている。
MQO ファイルは ReadMQOFile() メソッド、D3DX ファイルは ReadD3DXFile() メソッドで
形状データを入力することができる。なお、アニメーションに関しての機能が
\ref{sec:d3dxanimation} 節に記述してあるので、
そちらも合わせて参照してほしい。

ReadMQOFile() メソッドの引数構成は、以下のようになっている。
\\
\begin{screen}
\begin{verbatim}
  ReadMQOFile(String fileName, String objName, int matID, bool contFlg);
\end{verbatim}
\end{screen}
~ \\
「fileName」には MQO のファイル名、「objName」はファイル中のオブジェクト名を
入力する。「matID」は、特定のマテリアルを持つ面のみを抽出する場合はその ID を
指定する。全ての要素を読み込みたい場合は matID に -1 を入力する。

最後の「contFlg」はテクスチャ断絶のための設定である。
これは、テクスチャ座標が不連続な箇所に対し、形状の位相を断絶する操作を行うための
ものである。これを true にした場合断絶操作が行われ、テクスチャ座標が不連続な箇所が
幾何的にも不連続に表示されるようになる。ほとんどの場合、この操作を行った場合の方が
より適した描画となる。注意しなければならないのは、この断絶操作によって
MQOデータ中の位相構造とは異なる位相状態が内部で形成されることである。
そのため、頂点、稜線、面といった位相要素は MQO データよりも若干増加する。

なお、「matID」と「contFlg」はそれぞれ「-1」と「true」というデフォルト引数が
設定されており、このままで良いのであれば省略可能である。

ReadD3DXFile() メソッドの仕様に関しては、\ref{subsec:d3dxread} 節での内容と同じであるので、
そちらを参照してほしい。
また、\ref{subsec:mqodata} 節と同様の用途として、ReadMQOData() メソッドも利用できる。
引数の仕様は最初の引数が Byte 型配列になる以外は
上記 ReadMQOFile() メソッドとと同様である。

以下の例は MQO ファイルからの読み込みのサンプルで、
テクスチャ用画像ファイル名(Windows Bitmap 形式)が「sample.bmp」、
MQO ファイル名が「sample.mqo」、
ファイル中のオブジェクト名が「obj1」であることを想定している。
\\
\begin{breakbox}
\begin{verbatim}
        var texture = new fk_IFSTexture();

        if(texture.ReadBMP("sample.bmp") == false)
        {
            Console.WriteLine("Image File Read Error");
        }

        if(texture.ReadMQOFile("sample.mqo", "obj1") == false)
        {
            Console.WriteLine("Shape File Read Error");
        }
\end{verbatim}
\end{breakbox} ~

ちなみに、ここで読み込んだ形状データは \ref{sec:smooth} 節で
述べているスムースシェーディングの制御に対応している。

fk\_IFSTexture クラスが持つその他のメソッドとして、以下のようなものがある。
\begin{description}
\item[void Init()] ~ \\
	テクスチャデータ及び形状データの初期化を行うメソッド。\\

\item[fk\_TexCoord GetTextureCoord(int triID, int vID)] ~ \\
	三角形 ID が triID、頂点 ID が vID である頂点に
	設定されているテクスチャ座標を返すメソッド。\\

\item[void SetTextureCoord(int triID, int vID, fk\_TexCoord coord)] ~ \\
	三角形 ID が triID、頂点 ID が vID である頂点に
	coord をテクスチャ座標として設定する。\\

\item[fk\_IndexFaceSet IFS)] ~ \\
	fk\_IFSTexture クラスは、形状データとして
	内部では fk\_IndexFaceSet クラスによる変数を保持しており、
	その中に形状データを格納している。このプロパティは、
	そのインスタンスを取得するものである。
	頂点の移動などは、このインスタンスを介して fk\_IndexFaceSet の
	機能を用いて可能となる。
\end{description}

\subsection{メッシュテクスチャ}
最後に、「メッシュテクスチャ」を紹介する。メッシュテクスチャは、
前述した三角形テクスチャを複数枚同時に定義できる機能を持っている。
これは、「fk\_MeshTexture」というクラスを用いて実現できる。
このクラスは、前述の \ref{subsec:ifstexture}節で述べた
IFSテクスチャとよく似ているが、以下のような点が異なっている。
これらの性質を踏まえて、両方を使い分けてほしい。

\begin{table}[H]
\caption{IFSテクスチャとメッシュテクスチャの比較}
\label{tbl:meshtexture}
\begin{center}
\begin{tabular}{|l||l|l|}
\hline
項目 & IFSテクスチャ & メッシュテクスチャ \\ \hline \hline
形状生成 & ファイル入力のみ & ファイル入力とプログラムによる動的生成 \\ \hline
描画速度 & 高速 & IFSテクスチャより若干低速 \\ \hline
テクスチャ断絶 & 対応 & 非対応 \\ \hline
D3DX アニメーション & 対応 & 非対応 \\ \hline
\end{tabular}
\end{center}
\end{table}

使い方は、まず生成する三角形テクスチャの枚数を TriNum プロパティに対して設定する。
その後、fk\_TriTexture と同様に SetTextureCoord() メソッドで各頂点の
テクスチャ座標を、SetVertexPos() で空間上の位置座標を入力していくが、
それぞれのメソッドの引数の最初に三角形の ID を入力するところだけが異なっている。
\\
\begin{breakbox}
\begin{verbatim}
        var texture = new fk_MeshTexture();

        texture.TriNum = 2;
        texture.SetTextureCoord(0, 0, 0.0, 0.0);
        texture.SetTextureCoord(0, 1, 1.0, 0.0);
        texture.SetTextureCoord(0, 2, 0.5, 0.5);
        texture.SetTextureCoord(1, 0, 0.0, 0.0);
        texture.SetTextureCoord(1, 1, 0.5, 0.5);
        texture.SetTextureCoord(1, 2, 0.0, 1.0);
        texture.SetVertexPos(0, 0, 0.0, 0.0, 0.0);
        texture.SetVertexPos(0, 1, 50.0, 0.0, 0.0);
        texture.SetVertexPos(0, 2, 20.0, 30.0, 0.0);
        texture.SetVertexPos(1, 0, 0.0, 0.0, 0.0);
        texture.SetVertexPos(1, 1, 20.0, 30.0, 0.0);
        texture.SetVertexPos(1, 2, 0.0, 50.0, 0.0);
\end{verbatim}
\end{breakbox}
~ \\
別の生成方法として、
Metasequoia によって生成したテクスチャ付きの MQO ファイルを
読み込むことも可能である。以下のように、ReadMQOFile() メソッドを利用する。
例では、テクスチャ用画像ファイル名が「sample.bmp」、
MQO ファイル名が「sample.mqo」、
ファイル中のオブジェクト名が「obj1」と想定している。
\\
\begin{breakbox}
\begin{verbatim}
        fk_MeshTexture  texture;

        if(texture.ReadBMP("sample.bmp") == false)
        {
            Console.WriteLine("File Read Error");
        }

        if(texture.ReadMQOFile("sample.mqo", "obj1") == false)
        {
            Console.WriteLine("File Read Error");
        }
\end{verbatim}
\end{breakbox}
~ \\
また、fk\_MeshTexture クラスは複数のテクスチャ三角形平面によって構成されることに
なるが、PutIndexFaceSet メソッドを用いることによりその形状を
fk\_IndexFaceSet 型の形状として出力することが可能である。
\\
\begin{breakbox}
\begin{verbatim}
        var texture = new fk_MeshTexture();
        var ifset = new fk_IndexFaceSet();

        texture.PutIndexFaceSet(ifset);
\end{verbatim}
\end{breakbox}

\subsection{テクスチャのレンダリング品質設定}
矩形テクスチャ、三角形テクスチャ、IFSテクスチャ、メッシュテクスチャの全てにおいて、
レンダリングの品質を設定することができる。やりかたは、
以下のように RendMode プロパティで設定を行えばよい。
\\
\begin{screen}
\begin{verbatim}
        var texture = new fk_RectTexture();
                :
                :
        texture.RendMode = fk_TexRendMode.SMOOTH;
\end{verbatim}
\end{screen}
~ \\
モードは、通常モードである「fk\_TexRendMode.NORMAL」と、
アンチエイリアシング処理で高品質なレンダリングを行う
「fk\_TexRendMode.SMOOTH」が指定できる。デフォルトでは
通常モード(fk\_TexRendMode.NORMAL)となっている。

上記の例は矩形テクスチャで行っているが、
fk\_RectTexture, fk\_TriTexture, fk\_IFSTexture のいずれの型でも
同様に利用できる。

\section{画像処理用クラス} \label{sec:image}
第\ref{sec:texture}節で述べたテクスチャは、画像をファイルから読み込むことを
前提としていたが、用途によってはプログラム中で画像を生成し、
それをテクスチャマッピングするという場合もある。そのような場合、
fk\_Image というクラスを用いて画像を生成することが可能である。
fk\_RectTexture、fk\_TriTexture、fk\_IFSTexture、fk\_MeshTexture にはそれぞれ
Image というプロパティが用意されており、fk\_Image クラスの
変数をこのプロパティに設定することによって、それぞれのテクスチャに
画像情報が反映されるようになっている。

以下のプログラムは、赤から青へのグラデーションを表す画像を生成し、
fk\_RectTexture に画像情報を反映させるプログラムである。
\\
\begin{breakbox}
\begin{verbatim}
        var texture = new fk_RectTexture();
        var image = new fk_Image();

        // 画像サイズを 256x256 に設定
        image.NewImage(256, 256);

        // 各画素に色を設定
        for(int i = 0; i < 256; i++) {
            for(int j = 0; j < 256; j++) {
                Image.SetRGB(256-j, 0, j);
            }
        }

        // テクスチャに色を設定
        texture.Image = image;
\end{verbatim}
\end{breakbox}
~ \\
fk\_Image クラスの主要なメソッドやプロパティを以下に羅列する。
\begin{description}
\item[void Init()] ~ \\
	画像情報を初期化するメソッド。\\

\item[bool ReadBMP(String fileName)] ~ \\
	ファイル名が fileName である Windows Bitmap 形式の
	画像ファイルを読み込むメソッド。成功すれば true を、
	失敗すれば false を返す。\\

\item[bool ReadPNG(String fileName)] ~ \\
	ファイル名が fileName である PNG 形式の
	画像ファイルを読み込むメソッド。成功すれば true を、
	失敗すれば false を返す。\\

\item[bool ReadJPG(String fileName)] ~ \\
	ファイル名が fileName である JPEG 形式の
	画像ファイルを読み込むメソッド。成功すれば true を、
	失敗すれば false を返す。\\

\item[bool WriteBMP(String fileName, bool transFlag)] ~ \\
	現在格納されている画像情報を、Windows Bitmap 形式で
	ファイル名が fileName であるファイルに書き出すメソッド。
	transFlag を true にすると、透過情報を付加した
	32bit データとして出力し、false の場合通常の
	フルカラー 24bit 形式で出力する。
	書き出しに成功すれば true を、失敗すれば false を返す。\\

\item[bool WritePNG(String fileName, bool transFlag)] ~ \\
	現在格納されている画像情報を、PNG 形式で
	ファイル名が fileName であるファイルに書き出すメソッド。
	transFlag を true にすると、透過情報も合わせて出力する。
	false の場合は透過情報を削除したファイルを生成する。
	書き出しに成功すれば true を、失敗すれば false を返す。\\

\item[bool WriteJPG(String fileName, int quality)] ~ \\
	現在格納されている画像情報を、JPEG 形式で
	ファイル名が fileName であるファイルに書き出すメソッド。
	quality は 0 から 100 までの整数値を入力し、
	画像品質を設定する。
	数値が低いほど圧縮率は高いが画像品質は低くなる。
	数値が高いほど圧縮率は悪くなるが画像品質は良くなる。
	書き出しに成功すれば true を、失敗すれば false を返す。\\

\item[void NewImage(int w, int h)] ~ \\
	画像の大きさを横幅 w, 縦幅 h に設定するメソッド。これまでに
	保存されていた画像情報は失われる。\\

\item[void CopyImage(const fk\_Image image)] ~ \\
	image の画像情報をコピーするメソッド。\\

\item[void CopyImage(const fk\_Image image, int x, int y)] ~ \\
	現在の画像に対し、image の画像情報を
	左上が \((x, y)\) となる位置に上書きを行うメソッド。
	image はコピー先の中に完全に包含されている必要があり、
	はみ出てしまう場合には上書きは行われない。\\

\item[void SubImage(const fk\_Image image, int x, int y, int w, int h)] ~ \\
	元画像 image に対し、
	左上が \((x, y)\)、横幅 w, 縦幅 h となるような
	部分画像をコピーするメソッド。
	x, y, w, h に不適切な値が与えられた場合は、
	コピーを行わない。\\

\item[fk\_Dimension Size] ~ \\
	画像の縦横幅を得るプロパティ。\\

\item[int GetR(int x, int y)] ~ \\
	(x, y) の位置にある画素の赤要素の値を int 型で返すメソッド。\\

\item[int GetG(int x, int y)] ~ \\
	(x, y) の位置にある画素の緑要素の値を int 型で返すメソッド。\\

\item[int GetB(int x, int y)] ~ \\
	(x, y) の位置にある画素の青要素の値を int 型で返すメソッド。\\

\item[int GetA(int x, int y)] ~ \\
	(x, y) の位置にある画素の透明度要素の値を int 型で返すメソッド。\\

\item[bool SetRGBA(int x, int y, int r, int g, int b, int a)] ~ \\
	(x, y) の位置にある画素に対し、
	赤、緑、青、透明度をそれぞれ r, g, b, a に設定するメソッド。
	成功すれば true を、失敗すれば false を返す。\\

\item[bool SetRGB(int x, int y, int r, int g, int b)] ~ \\
	(x, y) の位置にある画素に対し、
	赤、緑、青をそれぞれ r, g, b に設定するメソッド。
	成功すれば true を、失敗すれば false を返す。\\

\item[bool SetR(int x, int y, int r)] ~ \\
	(x, y) の位置にある画素に対し、
	赤要素を r に設定するメソッド。
	成功すれば true を、失敗すれば false を返す。\\

\item[bool SetG(int x, int y, int g)] ~ \\
	(x, y) の位置にある画素に対し、
	緑要素を g に設定するメソッド。
	成功すれば true を、失敗すれば false を返す。\\

\item[bool SetB(int x, int y, int b)] ~ \\
	(x, y) の位置にある画素に対し、
	青要素を b に設定するメソッド。
	成功すれば true を、失敗すれば false を返す。\\

\item[bool SetA(int x, int y, int a)] ~ \\
	(x, y) の位置にある画素に対し、
	透明度要素を a に設定するメソッド。
	成功すれば true を、失敗すれば false を返す。\\

\item[void FillColor(fk\_Color col)] ~ \\
	画像中の全てのピクセルの色要素を col が表わす色に設定するメソッド。\\

\item[void FillColor(int r, int g, int b, int a)] ~ \\
	画像中の全てのピクセルの色要素を、r を赤、b を青、
	b を緑、a を透明度として設定するメソッド。\\
\end{description}

また、以下のようにすることで、
fk\_Colol 型によって画素色値の参照や設定を行うことができる。
\\
\begin{breakbox}
\begin{verbatim}
        var image = new fk_Image();

        // (x, y) = (100, 100) の画素色値を取得
        var color = image[100, 100];

        // (x, y) = (50, 70) の画素色値を取得
        image[50, 70] = new fk_Color(0.4, 0.5, 0.3);
\end{verbatim}
\end{breakbox}
