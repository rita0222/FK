\chapter{ベクトル、行列、四元数(クォータニオン)} \label{chap:vector} ~

この章では、ベクトル、行列、四元数といった数学的基本クラスの
利用方法を紹介する。

\section{3次元ベクトル}
この節では、fk\_Vector と呼ばれるベクトルや座標を司るクラスに関して述べる。
ベクトルは、単なる浮動小数点数が 3 つならんでいるという以上に
多くの意味を持つ。例えば、ベクトルは加減法、実数との積や内積、外積
といった多くの演算をほどこす必要が度々現われる。そのため、fk\_Vector 型は
大きく2つの役割を持っている。ひとつは、そのようなベクトルの演算を
プログラムの中で比較的容易に実現することをサポートすることであり、
もうひとつは形状やモデルの挙動を制御するための引数として扱う
ことである。この節では、特に前者について述べている。後者に関しては
\ref{chap:shape} 章 〜 \ref{chap:model} 章で順次説明していく。
この節での真価は、\ref{chap:sample} 章のサンプルで様々な形で現われている。

\subsection{ベクトルの生成・設定}
ベクトルの生成はいたって単純に行われる。fk\_Vector 型の変数を定義すればよい。
ただし fk\_Vector 型はクラス型であるので、new を用いる必要がある。
\\
\begin{screen}
\begin{verbatim}
        var vec = new fk_Vector();
\end{verbatim}
\end{screen}
~ \\
このとき、vec 変数は初期成分として (0, 0, 0) が代入される。
また、初期値を最初から代入することも可能である。
\\
\begin{screen}
\begin{verbatim}
        var vec = new fk_Vector(1.0, 1.0, -3.0);
\end{verbatim}
\end{screen}
~ \\
この記述は、vec に (1, 1, -3) を代入する。

もちろん、生成後の代入も可能である。次のように記述すればよい。
\\
\begin{screen}
\begin{verbatim}
        var vec1 = new fk_Vector();
        var vec2 = new fk_Vector();

        vec1.Set(1.0, 1.0, -3.0);
        vec2.Set(5.0, -2.5);
\end{verbatim}
\end{screen}
~ \\
Set メソッドが、vec1 や vec2 に値を代入してくれる。
vec2 のように引数が2個しか無い場合は、\(z\) 成分には自動的に
\(0\) が代入されるため、結果的に vec2 に \((5, -2.5, 0)\) が
代入されることになる。

また、単に
\\
\begin{screen}
\begin{verbatim}
        vec.Init();
\end{verbatim}
\end{screen}
~ \\
と記述した場合、vec はゼロベクトル (0, 0, 0) となる。

ベクトル中の x, y, z の各成分はすべて public メンバとなっているので、
直接参照や代入を行うことが可能である。たとえば、
\(z\) 要素が正か負かを調べたいときは、
\\
\begin{screen}
\begin{verbatim}
        if(vec.z < 0.0) {
                // 後処理
        }
\end{verbatim}
\end{screen}
~ \\
と書けばよい。代入に関しても、以下のような記述が可能である。
\\
\begin{screen}
\begin{verbatim}
        vec.x = tmpX;
        vec.y = tmpY;
        vec.z = tmpZ;
\end{verbatim}
\end{screen}
~ \\
ベクトルを配列として定義した場合も、もちろん各成分を直接扱うことが
可能である。以下に例を示す。
\\
\begin{screen}
\begin{verbatim}
        var vec = new fk_Vector[10];

        for(int i = 0; i < 10; i++) {
                vec[i] = new fk_Vector();
                vec[i].x = (double)i;
                vec[i].y = (double)-i;
                vec[i].z = 1.0;
        }
\end{verbatim}
\end{screen}
\subsection{比較演算}
fk\_Vector 型はインスタンス同士が持つ値が等しいかどうかを判定する 
Equals() を利用することができる。通常の int 型の変数の場合は、
(a == b)のようにして値を比較できるが、fk\_Vector 型はクラスであるため、
== を利用した場合はインスタンスが同一であるかの判定になってしまう。
このため、vecA と vecB が持つ値が等しいか否かを判定するには、
(vecA.Equals(vecB))と記述する必要がある。
判定結果が bool 型として返る点は、通常の比較演算と同様である。
このとき注意しなければならないのは、この比較演算ではある程度の
数値誤差を許していることである。具体的に述べると、もし vec1.y
と vec2.y、vec1.z と vec2.z の値がともに等しいとして、vec1.x
と vec2.x が \(10^{-14}\) 程度の差が存在しても、(vec1.Equals(vec2))
は真(true)を返すのである。現在、この許容誤差は \(10^{-12}\) に設定されている。
ちなみに、偽の場合は false を返す。
\subsection{単項演算子}
fk\_Vector 型は、単項演算子 `-' を持っている。これは、
ベクトルを反転したものを返すもので、例えば、
\\
\begin{screen}
\begin{verbatim}
        vec2 = -vec1;
\end{verbatim}
\end{screen}
~ \\
という記述は vec2 に vec1 を反転したものを代入する。この際、vec1 の
値そのものは変化しない。これは、通常の int 型や double 型の変数の場合
と同じ挙動であるといえる。
\subsection{二項演算子}
fk\_Vector の二項演算子は多様であり、どれも実践的なものである。
表\ref{tbl:fkVec1}
にそれらを羅列する。
\begin{table}[H]
\caption{fk\_Vector の二項演算子}
\label{tbl:fkVec1}
\begin{center}
\begin{tabular}{|c|c|c|c|}
\hline
演算子 & 形式 & 機能 & 返り値の型 \\ \hline \hline
\verb-+- & \(Vector \; + \; Vector\) & ベクトルの和 & fk\_Vector \\ \hline
\verb+-+ & \(Vector \; - \; Vector\) & ベクトルの差 & fk\_Vector \\ \hline
\verb+*+ & \(double \; * \; Vector\) & ベクトルの実数倍 & fk\_Vector \\ \hline
\verb+*+ & \(Vector \; * \; double\) & ベクトルの実数倍 & fk\_Vector \\ \hline
\verb+/+ & \(Vector \; / \; double\) & ベクトルの実数商 & fk\_Vector \\ \hline
\verb+*+ & \(Vector \; * \; Vector\) & ベクトルの内積 & double \\ \hline
\verb+^+ & \(Vector \; \hat{} \; Vector\) & ベクトルの外積 &
							fk\_Vector \\ \hline
\end{tabular}
\end{center}
\end{table}

\subsection{代入と演算子}
fk\_Vector 型はクラスであるため、代入処理ではコピーが行われず、
左辺値が右辺値と同一のインスタンスを参照するようになる。したがって、
\\
\begin{screen}
\begin{verbatim}
        var vec1 = new fk_Vector();
        var vec2 = new fk_Vector(1.0, 1.0, 1.0);

        vec1 = vec2;
        vec2.x = 3.0;
\end{verbatim}
\end{screen}
~ \\
という記述を行った場合、vec1 と vec2 ともに \(3.0, 1.0, 1.0\) という値を指すことになる。
fk\_Vector 型の値をコピーしたい時は、コピーコンストラクタによって明示的に指示する必要がある。
例えば、
\\
\begin{screen}
\begin{verbatim}
        var vec1 = new fk_Vector();
        var vec2 = new fk_Vector(1.0, 1.0, 1.0);

        vec1 = new fk_Vector(vec2);
        vec2.x = 3.0;
\end{verbatim}
\end{screen}
~ \\
このように記述した場合、vec1 は \(1.0, 1.0, 1.0\) という値のコピーを保持し、
vec2 の値は x 成分への代入によって \(3.0, 1.0, 1.0\) に変更される。

その他の代入演算子として、次のようなものが使用できる。効果は、
表\ref{tbl:fkVec2}の右に掲載された式と同様の働きである。なお、
効果で left は左辺、right は右辺を指す。
\begin{table}[H]
\caption{fk\_Vector の代入演算子}
\label{tbl:fkVec2}
\begin{center}
\begin{tabular}{|c|c|c|}
\hline
演算子 & 形式 & 効果 \\ \hline \hline
\verb-+=- & \(Vector\) \verb-+=- \(Vector\) &
	left = left + right \\ \hline
\verb+-=+ & \(Vector\) \verb+-=+ \(Vector\) &
	left = left - right \\ \hline
\verb+*=+ & \(Vector\) \verb+*=+ \(double\) &
	left = left * right \\ \hline
\verb+/=+ & \(Vector\) \verb+/=+ \(double\) &
	left = left / right \\ \hline
\end{tabular}
\end{center}
\end{table}

\subsection{ノルム(長さ)の算出}
ベクトルのノルム(長さ)を出力するメソッドとして Dist() が、
ノルムの2乗を指す Dist2() があり、
それぞれ double 型を返す。使用法は次のようなものである。
\\
\begin{screen}
\begin{verbatim}
        double l = vec1.Dist();
        double l2 = vec1.Dist2();
        Console.WriteLine("length = {0}", l);
        Console.WriteLine("length^2 = {0}", l2);
\end{verbatim}
\end{screen}
~ \\
処理速度は、Dist2() の方が Dist() よりも高速である。従って、ただ単に
ベクトルの長さを比較したいだけならば Dist2() を利用した方が効率的である。

\subsection{ベクトルの正規化}
正規化とは、零ベクトルでない任意のベクトル \(\mathbf{V}\) に対し、
\[
	\mathbf{V}' = \frac{\mathbf{V}}{|\mathbf{V}|}
\]
を求めることである。\(\mathbf{V}'\) は、結果的には \(\mathbf{V}\) と
方向が同一で長さが 1 のベクトルとなる。3次元中の様々な幾何計算や、
コンピュータグラフィックスの理論では、しばしば正規化されたベクトルを
用いることが多い。

fk\_Vector 型の変数に対し、自身を正規化(Normalize)するメソッドとして
Normalize() がある。
\\
\begin{screen}
\begin{verbatim}
        var vec1 = new fk_Vector(5.0, 4.0, 3.0);

        vec1.Normalize();
\end{verbatim}
\end{screen}
~ \\
という記述は、vec1 を正規化する。ちなみに、Normalize() メソッドは
bool 型を返し、true ならば正規化の成功、false ならば失敗を意味する。
失敗するのは、ベクトルが零ベクトル(つまり長さが 0)の場合に限られる。

\subsection{ベクトルの射影}
ベクトルを用いた理論では、射影と呼ばれる概念がよく用いられる。
射影とは、図 \ref{fig:vecproj} にあるようにあるベクトルに対して
別のベクトルを投影することである。ここで、図 \ref{fig:vecproj}
の \(\bP_{proj}\) を「\(\bP\) の \(\bQ\) に対する\textbf{射影ベクトル}」
と呼ぶ。

\myfig{fig:vecproj}{./Fig/VecProj.eps}{scale=1.5,clip}
	{ベクトルの射影}{0mm}

この射影ベクトルを求める方法として、fk\_Vector では
Proj() メソッドを利用することができる。
\\
\begin{screen}
\begin{verbatim}
        fk_Vector   P, Q, P_proj; // P, Qに関してはこの後 new して何らかの値が入るものとする
                :
        P_proj = P.Proj(Q);
\end{verbatim}
\end{screen}
~ \\	
また、投影の際の垂直成分 (図 \ref{fig:vecproj} 中での \(\bP_{perp}\))
を求めるときは、Perp() というメソッドを利用する。
\\
\begin{screen}
\begin{verbatim}
        P_perp = P.Perp(Q);
\end{verbatim}
\end{screen}
\section{行列}
この節は、FK System の持つ行列演算に関して紹介する。
行列を用いることによってよりプログラムの記述が洗練されたり、あるいは
高速な実行を可能にする。なお、この節の解説は
線形代数、特に1次変換の知識があることを前提としている。
初学者は、必要となるまで読み飛ばしてもらっても差し支えない。

\subsection{FKにおける行列系}
FK システムでは「MV 行列系」という行列系を
採用している。これは、行列とベクトルの演算を以下のように規定するものである。
\begin{itemize}
 \item ベクトルは、通常列ベクトルとして扱う。
 \item 行列とベクトルの積は、通常左側に行列、右側にベクトルを置くものとする。
 \item 行列の積が生成された場合に、
	ベクトルに先に作用するのは右項の行列である。
\end{itemize}
一般に、行列を扱う数学書においては、上記の MV 行列系を前提として
記述されているものがほとんどであり、それらの理論を参考にするのに
都合が良いと言える。一方で、3DCGの開発システムにおいて、
MV行列系とは逆の「VM行列系」を採用しているものもあり
\footnote{例えば、マイクロソフト社の「Direct3D」は
行列系として VM 系を採用している。}、
それらの理論や実装を参考にする際には注意が必要である。

\subsection{行列の生成}
一般的に、3DCG の世界で頻繁に用いられる行列は 4 行 4 列の
正方行列である。これは、4 行 4 列であることによって、
3次元座標系の回転移動、平行移動、線形拡大縮小を全て表現できるからである。
FK System において、4 行 4 列の正方行列は fk\_Matrix という型で提供されている。

fk\_Matrix 型は、定義時には単位行列(零行列ではない)が初期値として設定される。
fk\_Matrix は、fk\_Vector のように初期値設定の手段を持たない。
そのかわり、多様な設定方法が存在する。表\ref{tbl:fkMat1}はそれをまとめた
ものである。

\begin{table}[H]
\caption{fk\_Matrix の初期値設定用メソッド}
\label{tbl:fkMat1}
\begin{center}
\begin{tabular}{|c|c|c|}
\hline
メソッド名と引数 & 引数の意味 & 効果 \\ \hline \hline
MakeRot(double, fk\_Axis) & 角度数と軸 &
        回転行列の生成 \\ \hline
MakeTrans(double, double, double) &
        x, y, zに対応する実数値 & 平行移動行列の生成 \\ \hline
MakeTrans(fk\_Vector) & ベクトル & 上記に同じ \\ \hline
MakeScale(double, double, double) &
        x, y, zに対応する実数値 & 拡大縮小行列の生成 \\ \hline
MakeScale(fk\_Vector) & ベクトル & 上記に同じ \\ \hline
MakeEuler(double, double, double) & オイラー角に相当する実数値 &
        オイラー回転行列の生成 \\ \hline
MakeEuler(fk\_Angle) & オイラー角 & 上記に同じ \\ \hline
\end{tabular}
\end{center}
\end{table}

MakeRot メソッドは、2つ目の引数に \(x\) 軸を表す fk\_Axis.X、
\(y\) 軸を表す fk\_Axis.Y、\(z\) 軸を表す fk\_Axis.Z の
いずれかをとる。最初の引数である実数値は弧度法
(rad ラジアン)として扱われる。\(180^\circ = \pi \: rad\) である。
.NET 環境では \(\pi\) を表す定義値として Math.PI を提供している。
したがって、たとえば 90 度は \verb+Math.PI/2.0+ と表せばよい。

MakeEuler メソッドは引数にそれぞれ順に heading 角、pitch 角、bank 角を
与える。単位は MakeRot メソッドと同様に弧度法である。fk\_Angle 型は、
オイラー角を表わすクラスで、heading 角、pitch 角、bank 角にあたる
メンバはそれぞれ h, p, b となっており、public アクセスが可能である。
\subsection{比較演算}
fk\_Matrix 型も、fk\_Vector 型のような Equals() メソッドを持ち合わせている。
これらは、やはり許容誤差を持って判定される。
\subsection{逆行列演算}
fk\_Matrix 型は、GetInverse() メソッドを持っている。これは逆行列を
返すものであり、自身に変化は起こさない。
以下のプログラムは、行列 A の逆行列を B に代入するものである。
\\
\begin{screen}
\begin{verbatim}
        fk_Matrix   A, B; // A にはこの後 new して何らかの値が入るものとする
                :
        B = A.GetInverse();
\end{verbatim}
\end{screen}
\subsection{二項演算子}
fk\_Matrix 型の二項演算子は表\ref{tbl:fkMat2}の通りのものが用意されている。
\begin{table}[H]
\caption{fk\_Matrix の二項演算子}
\label{tbl:fkMat2}
\begin{center}
\begin{tabular}{|c|c|c|c|}
\hline
演算子 & 形式 & 機能 & 返り値の型 \\ \hline \hline
\verb-+- & \(Matrix \; + \; Matrix\) & 行列の和 & fk\_Matrix \\ \hline
\verb+-+ & \(Matrix \; - \; Matrix\) & 行列の差 & fk\_Matrix \\ \hline
\verb+*+ & \(Matrix \; * \; Matrix\) & 行列の積 & fk\_Matrix \\ \hline
\verb+*+ & \(Matrix \; * \; Vector\) & ベクトルと行列の積 & fk\_Vector \\ \hline
\end{tabular}
\end{center}
\end{table}

\subsection{代入演算子}
fk\_Matrix 型においても、代入処理における挙動は fk\_Vector 型とまったく同様である。
すなわち、変数同士の代入は参照先を同一にするのみで、
明示的にコピーを行いたい場合はコピーコンストラクタを利用することとなる。
その他の代入演算子も用意されており、それは表\ref{tbl:fkMat3}のようなものである。
記述法は、表\ref{tbl:fkVec2}と同様である。

\begin{table}[H]
\caption{fk\_Matrix の代入演算子}
\label{tbl:fkMat3}
\begin{center}
\begin{tabular}{|c|c|c|}
\hline
演算子 & 形式 & 効果 \\ \hline \hline
\verb-+=- & \(Matrix\) \verb-+=- \(Matrix\) & left = left + right \\ \hline
\verb+-=+ & \(Matrix\) \verb+-=+ \(Matrix\) & left = left - right \\ \hline
\verb+*=+ & \(Matrix\) \verb+*=+ \(Matrix\) & left = left * right \\ \hline
\verb+*=+ & \(Matrix\) \verb+*=+ \(Vector\) & left = right * left \\ \hline
\end{tabular}
\end{center}
\end{table}
\subsection{各成分へのアクセス}
行列成分へのアクセスは、配列演算子を用いる。これは2次元で定義されて
おり、1次元目が行を、2次元目が列を表している。行と列はカンマで区切って指定する。\\
\begin{screen}
\begin{verbatim}
        var mat = new fk_Matrix();

        mat.MakeEuler(Math.PI/2.0, Math.PI/4.0, Math.PI/6.0);
        Console.WriteLine("mat[1][0] = {0}", mat[1,0]);
\end{verbatim}
\end{screen}
\subsection{その他のメソッド}
以下に、fk\_Matrix で用いられる主要なメソッドを紹介する。

\subsubsection*{void Init()}
自身を単位行列にする。

\subsubsection*{void Set(int r, int c, double a)}
行番号が r、列番号が c に対応する成分の値を a に設定する。

\subsubsection*{void SetRow(int r, fk\_Vector v)}
\subsubsection*{void SetRow(int r, fk\_HVector v)}
行番号が r である行ベクトルを v の各成分値に設定する。

\subsubsection*{void SetCol(int c, fk\_Vector v)}
\subsubsection*{void SetCol(int c, fk\_HVector v)}
列番号が c である列ベクトルを v に設定する。

\subsubsection*{fk\_HVector GetRow(int r)}
行番号が r である行ベクトルを取得する。

\subsubsection*{fk\_HVector GetCol(int c)}
列番号が c である列ベクトルを取得する。

\subsubsection*{bool IsSingular(void)}
自身が特異行列であるかどうかを判定する。
特異行列とは、逆行列が存在しない行列のことである。
特異行列である場合は true を、そうでない場合は false を返す。

\subsubsection*{bool IsRegular(void)}
自身が正則行列であるかどうかを判定する。
正則行列とは、逆行列が存在する行列のことである。
(つまり、「非正則行列」と「特異行列」はまったく同じ意味になる。)
正則行列である場合は true を、そうでない場合は false を返す。

\subsubsection*{bool Inverse(void)}
自身が正則行列であった場合、
自身を逆行列と入れ替えて true を返す。
特異行列であった場合は、false を返し自身は変化しない。

\subsubsection*{void Negate(void)}
自身を転置する。転置とは、
行列の行と列を入れ替える操作のことである。

\subsection{4次元ベクトル}
fk\_Matrix は 4 行 4 列の正方行列であるため、本来であれば、
fk\_Matrix 型に対応するベクトルは 4 次元でなければならない。
しかし、fk\_Vector 型は 3 次元であるため、そのままでは
積演算ができないことになる。そのため、FK では
4次元ベクトル用クラスとして fk\_HVector クラスがあり、
実際の行列演算は fk\_HVector を用いて行うという仕組みになっている。

fk\_HVector クラスは fk\_Vector クラスの派生クラスであり、
4次元目の成分「w」を持つ。この成分は double 型の public メンバとして
定義されており、自由にアクセスすることができる。
座標変換においては、w 成分は同次座標を表わすことを想定している。
同次座標は通常 1 で固定されるが、「射影幾何学」と呼ばれる数学分野に
おいては、この同次座標を操作する理論もある。

fk\_Matrix 型と fk\_Vector 型の積演算は、実際には以下のような
処理が FK 内部で行われる。
\begin{enumerate}
 \item まず、fk\_Vector 型変数に対し fk\_HVector 型に暗黙の型変換が行われる。
 \item 変換後の fk\_HVector オブジェクトと fk\_Matrix による積を算出し、
	その結果が fk\_HVector 型として返される。
 \item 返り値である fk\_HVector オブジェクトから、暗黙の型変換によって
	fk\_Vector 型変数に代入される。
\end{enumerate}
この仕組みにより、FK の利用者が通常の行列演算において
fk\_HVector の存在を意識する必要はない。
しかし、あえて fk\_HVector を利用することによる利点もある。

4行4列の行列は、回転等による姿勢変換と平行移動変換を、
行列同士の積演算を行うことによって同時に内包することができる。
物体の位置や形状頂点などの位置ベクトルに関しては、同次座標が1であることに
よってそのまま変換が可能であるが、モデルの姿勢等の方向ベクトルに関しては
同次座標が1のまま変換を行うと間違った結果を生じてしまう。
これは、方向ベクトルの変換に関しては姿勢変換のみが適用されるべきであるのに
対し、平行移動も適用してしまうからである。

このようなとき、同次座標を 0 に設定したベクトルで処理を行うとよい。
それにより、行列中の平行移動用成分(4列目)が結果に作用しなくなり、
姿勢変換のみによる結果を得ることが可能となる。
このような処理を実現するためには、単純に w メンバに 0 を代入するだけで
可能であるが、一応専用のメソッドも準備されている。
IsPos() メソッドは同次座標を 1 に設定し、平行移動変換を有効とする。
IsVec() メソッドは同次座標を 0 に設定し、平行移動変換を無効とする。

上記のような処理をプログラム中で実現したい場合は、fk\_HVector クラスを
利用するとよいだろう。

\section{四元数(クォータニオン)}
四元数(クォータニオン)は、
3種類の虚数単位 \(i, j, k\) と
4個の実数 \(s, x, y, z\) を
用いて
\[
	\bq = s + xi + yj + zk
\]
という形式で表現される数のことであり、
発見者のハミルトンにちなんで「ハミルトン数」と呼ばれることもある。
この節では、FK 中で四元数を表わすクラス「fk\_Quaternion」の利用方法を
述べる。ただし、四元数の数学的定義や、オイラー角、行列と比較した
長所と短所に関してはここでは扱わない。適宜参考書を参照されたい。

なお、FKにおける四元数と行列、オイラー角に関する関係は
\begin{itemize}
 \item MV行列系。
 \item 右手座標系。
\end{itemize}
を満たすことを前提に構築されている。

\subsection{四元数クラスのメンバ構成}
四元数の 3 種類の虚数単位をそれぞれ \(i, j, k\) とし、
4 個の実数によって \(\bq = s + xi + yj + zk\) で表わされるとする。
このとき、実数部である \(s\) を\textbf{スカラー部}、虚数部である
\(xi + yj + zk\) を\textbf{ベクトル部}と呼ぶ。これは、
四元数ではしばしば虚数部係数をベクトル \((x, y, z)\) として扱うと、
都合の良いことが多々あるためである。

これにならい、四元数を表わすクラス fk\_Quaternion では、
四元数を 1 個の double 型実数 \verb+s+ と、fk\_Vector 型の
メンバ \verb+v+ によって表わしている。つまり、fk\_Quaternion 型の
変数を \verb+q+ とした場合、
\begin{center}
q.s, \quad q.v.x, \quad q.v.y, \quad q.v.z
\end{center}
として各成分にアクセスできることになる。これらは全て public メンバと
なっている。

\subsection{四元数の生成と設定}
fk\_Quaternion クラスは、デフォルトコンストラクタとして何も引数をとらない
コンストラクタがある。この場合、スカラー部である \verb+s+ が 1、
ベクトル部に零ベクトルが設定される。

その他にも、fk\_Quaternion には 2 種類のコンストラクタがある。
まず、4 個の実数を引数にとるもので、これはそれぞれスカラー部、
そしてベクトル部の \(x, y, z\) 成分に対応する。
もう1つのコンストラクタは実数 1 個と fk\_Vector 1 個をとるもので、
やはり同様にスカラー部とベクトル部に対応する。

それぞれの書式例は、以下のようなものになる。
\\
\begin{screen}
\begin{verbatim}
        var v = new fk_Vector(0.2, 0.5, 1.0);

        var q1 = new fk_Quaternion(0.3, 0.4, 2.0, -2.0);
        var q2 = new fk_Quaternion(0.5, v);
\end{verbatim}
\end{screen}

設定用のメソッドとしては、以下のようなものがある。

\subsubsection*{void init(void)}
初期化のためのメソッドで、デフォルトコンストラクタと同様にスカラー部が 1、
ベクトル部に零ベクトルが設定される。

\subsubsection*{void set(double t, const fk\_Vector \&v)}
設定のためのメソッドで、t がスカラー部、v がベクトル部に対応している。

\subsubsection*{void set(double s, double x, double y, double z)}
設定のためのメソッドで、四元数の各成分が順に対応している。

\subsubsection*{void setRotate(double theta, const fk\_Vector \&v)}
角度を theta、回転軸を v とするような回転変換を表わす四元数を生成する。
実際に四元数に代入される値は、theta を \(\theta\)、
v を \(\bV\) としたとき、
スカラー部が \(\cos\frac{\theta}{2}\)、
ベクトル部が \(\frac{\bV}{|\bV|}\sin\frac{\theta}{2}\) となる。

\subsubsection*{void setRotate(double theta, double x, double y, double z)}
角度を theta, 回転軸を \((x, y, z)\) とするような
回転変換を表わす四元数を生成する。
ベクトル部の引数が実数となる以外は、前述の setRotate と同様である。

\subsection{比較演算子}
fk\_Quaternion 型は、比較演算子として `\verb+==+' と `\verb+!=+' を
利用できる。fk\_Vector と同様に、ある程度の数値誤差を許している。

\subsection{単項演算子}
fk\_Quaternion 型では、以下の表\ref{tbl:fkQ1}に挙げる
3種類の単項演算子をサポートする。
\begin{table}[H]
\caption{fk\_Quaternion の単項演算子}
\label{tbl:fkQ1}
\begin{center}
\begin{tabular}{|c|c|}
\hline
演算子 & 効果 \\ \hline \hline
\verb+-+ & 符合転換を返す。 \\ \hline
\verb+~+ & 共役を返す。\\ \hline
\verb+!+ & 逆元を返す。\\ \hline
\end{tabular}
\end{center}
\end{table}
元となる四元数を \(\bq = s + xi + yj + zk\) とすると、
符合転換 \(-\bq\)、共役 \(\overline{\bq}\)、逆元 \(\bq^{-1}\) は
それぞれ以下の式によって求められる。
\begin{eqnarray*}
	-\bq &=& -s - xi - yj - zk \\
	\overline{\bq} &=& s - xi - yj - zk \\
	\bq^{-1} &=& \dfrac{\overline{\bq}}{|\bq|^2}
\end{eqnarray*}
なお、全ての単項演算子は自身には変化を及ぼさない。
\subsection{二項演算子}
fk\_Quaternion 型がサポートする二項演算子を、
表\ref{tbl:fkQ2}に羅列する。
\begin{table}[H]
\caption{fk\_Quaternion の二項演算子}
\label{tbl:fkQ2}
\begin{center}
\begin{tabular}{|c|c|c|c|}
\hline
演算子 & 形式 & 機能 & 返り値の型 \\ \hline \hline
\verb-+- & \(Quaternion \; + \; Quaternion\) &
	四元数の和 & fk\_Quaternion \\ \hline
\verb+-+ & \(Quaternion \; - \; Quaternion\) &
	四元数の差 & fk\_Quaternion \\ \hline
\verb+*+ & \(Quaternion \; * \; Quaternion\) &
	四元数の積 & fk\_Quaternion \\ \hline
\verb+*+ & \(double \; * \; Quaternion\) &
	四元数の実数倍 & fk\_Quaternion \\ \hline
\verb+*+ & \(Quaternion \; * \; double\) &
	四元数の実数倍 & fk\_Quaternion \\ \hline
\verb+/+ & \(Quaternion \; / \; double\) &
	四元数の実数商 & fk\_Quaternion \\ \hline
\verb+^+ & \(Quaternion \;\; \hat{} \;\; Quaternion\) &
	四元数の内積 & double \\ \hline
\verb+*+ & \(Quaternion \; * \; Vector\) &
	四元数によるベクトル変換 & fk\_Vector \\ \hline
\end{tabular}
\end{center}
\end{table}
この中で、内積値は \(\bq_1 = s_1 + x_1i + y_1j + z_1k, \;
\bq_2 = s_2 + x_2i + y_2j + z_2k\) としたとき、
\[
	\bq_1\cdot\bq_2 = s_1s_2 + x_1x_2 + y_1y_2 + z_1z_2
\]
という式によって求められる実数値のことである。また、ベクトル変換というのは
四元数 \(\bq\) と 3 次元ベクトル \(\bV\) に対し、
\[
	\bV' = \bq\bV\bq^{-1}
\]
という演算を施すことである。このとき、\(\bq\) が回転変換を表わす場合に、
\(\bV'\) は \(\bV\) を回転したベクトルとなる。
この演算は、行列による回転演算と比較して若干高速であることが知られている。

\subsection{代入演算子}
fk\_Quaternion 型の代入演算子 ` = ' は、fk\_Vector の場合と同様に
値のコピーが行われ、実体は別物となる。その他の代入演算子を
表\ref{tbl:fkQ3}に列挙する。
\begin{table}[H]
\caption{fk\_Quaternion の代入演算子}
\label{tbl:fkQ3}
\begin{center}
\begin{tabular}{|c|c|c|}
\hline
演算子 & 形式 & 効果 \\ \hline \hline
\verb-+=- & \(Quaternion\) \verb-+=- \(Quaternion\) &
	left = left + right \\ \hline
\verb+-=+ & \(Quaternion\) \verb+-=+ \(Quaternion\) &
	left = left - right \\ \hline
\verb+*=+ & \(Quaternion\) \verb+*=+ \(double\) &
	left = left * right \\ \hline
\verb+/=+ & \(Quaternion\) \verb+/=+ \(double\) &
	left = left / right \\ \hline
\verb+*=+ & \(Quaternion\) \verb+*=+ \(Quaternion\) &
	left = left * right \\ \hline
\end{tabular}
\end{center}
\end{table}
\subsection{オイラー角との相互変換}
四元数は任意軸による回転変換を表わすが、これはすなわちオイラー角と
同義の情報を持つことを意味する。fk\_Quaternion には、fk\_Angle と
相互変換を行うためのメソッドが用意されている。以下にその仕様を述べる。

\subsubsection*{void makeEuler(const fk\_Angle angle)}
angle が示すオイラー角と同義の四元数を設定する。

\subsubsection*{void makeEuler(double h, double p, double b)}
h をヘディング角、p をピッチ角、b をバンク角とするオイラー角と
同義の四元数を設定する。

\subsubsection*{fk\_Angle getEuler(void)}
同義となるオイラー角を返す。

\subsection{各種メソッド}
fk\_Quaternion には、これまで挙げた他にも以下のようなメソッドがある。
なお、文中では四元数自身を \(\bq = s + xi + yj + zk\) と想定する。

\subsubsection*{double norm(void)}
ノルム \(|\bq|^2 = s^2 + x^2 + y^2 + z^2\) を返す。

\subsubsection*{double abs(void)}
絶対値 \(|\bq| = \sqrt{s^2 + x^2 + y^2 + z^2}\) を返す。

\subsubsection*{bool normalize(void)}
自身の正規化四元数 \(\frac{\bq}{|\bq|}\) を求め、自身に上書きし true を返す。
ただし、成分が全て 0 であった場合は値を変更せずに false を返す。

\subsubsection*{bool inverse(void)}
自身の逆元 \(\bq^{-1}\)を求め、自身に上書きし true を返す。
ただし、成分が全て 0 であった場合は値を変更せずに false を返す。

\subsubsection*{void conj(void)}
自身の共役 \(\overline{\bq}\)を求め、自身に上書きする。

\subsubsection*{fk\_Matrix conv(void)}
自身が表わす回転変換と同義の行列を返す。

\subsection{補間メソッド}
四元数の最大の特徴は姿勢の補間である。あるオイラー角から別のオイラー角への
変化をスムーズに実現することは、オイラー角や行列のみを用いる場合は
難解であるが、四元数はこういった問題を解決するのに適している。

補間四元数を求めるために、FK では以下の2種類のメソッドが用意されている。

\subsubsection*{fk\_Quaternion \quad fk\_Math::quatInterLinear(
fk\_Quaternion q1, fk\_Quaternion q2, double t)}
q1 と q2 に対し、単純線形補間を行った結果を返す。t は 0 から 1 までの
パラメータで、0 の場合 q1、1 の場合 q2 と完全に一致する。
補間処理は以下の式に基づく。
\[
	\bq(t) = (1 - t)\bq_1 + t\bq_2
\]

\subsubsection*{fk\_Quaternion \quad fk\_Math::quatInterSphere(
fk\_Quaternion q1, fk\_Quaternion q2, double t)}
q1 と q2 に対し、球面線形補間を行った結果を返す。t は 0 から 1 までの
パラメータで、0 の場合 q1、1 の場合 q2 と完全に一致する。
補間処理は以下の式に基づく。
\[
	\bq(t) = \frac{\sin((1-t)\theta)}{\sin\theta}\bq_1 +
	       \frac{\sin(t\theta)}{\sin\theta}\bq_2
	       \qquad (\theta = \cos^{-1}(\bq_1\cdot\bq_2))
\]

演算そのものは単純線形補間の方が高速である。
しかし単純線形補間では、状況によっては不安定な結果を示す場合がある。
このような現象が許容できない場合には、球面線形補間による処理が有効である。
