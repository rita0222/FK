\chapter{サウンドの再生}
\section{音関係のセットアップ} \label{sec:10-setup}
\subsection{事前のセットアップ}
\verb+C:\FK_VC13\redist+ にある「oalinst.exe」という名前の
	インストーラを実行してください。これは1回やっておけば
	Windowsを再インストールしない限りは今後する必要はありません。
\subsection{プロジェクト毎のセットアップ}
音を利用するプロジェクトを作成した際には、
ビルド後の \verb+_exe+ フォルダに対し、
\verb+C:\FK_VC13\bin+ にある以下のファイルをコピーしてください。
\begin{itemize}
 \item libogg.dll
 \item libvorbis.dll
 \item libvorbisfile.dll
\end{itemize}

\section{BGMの再生} \label{sec:10-ogg}
\begin{itembox}[l]{Oggファイルの再生}
\begin{small}
\begin{verbatim}
 1: #include <FK/FK.h>
 2: #include <FK/Audio.h>
 3: 
 4: int main(int, char **)
 5: {
 6:     fk_AppWindow    window;
 7:     fk_Block        block(1.0, 1.0, 1.0);
 8:     fk_Model        blockModel;
 9:     fk_Vector       origin(0.0, 0.0, 0.0);
10:     fk_AudioStream  bgm;
11:     double          volume;
12: 
13:     if(bgm.open("epoq.ogg") == false) {
14:         fl_alert("BGM Open Error");
15:     }
16:     bgm.setLoopMode(true);
17: 
18:     blockModel.setShape(&block);
19:     blockModel.glMoveTo(3.0, 3.0, 0.0);
20:     blockModel.setMaterial(Yellow);
21:     window.entry(&blockModel);
22: 
23:     window.setCameraPos(0.0, 1.0, 20.0);
24:     window.setCameraFocus(0.0, 1.0, 0.0);
25: 
26:     window.setSize(800, 600);
27:     window.setBGColor(0.6, 0.7, 0.8);
28:     window.open();
29:     window.showGuide(FK_GRID_XZ);
30: 
31:     volume = 0.5;
32:     while(window.update() == true) {
33:         blockModel.glRotateWithVec(origin, fk_Y, FK_PI/360.0);
34: 
35:         if(window.getKeyStatus('Z') == FK_SW_DOWN && volume < 1.0) {
36:             volume += 0.1;
37:         }
38:         if(window.getKeyStatus('X') == FK_SW_DOWN && volume > 0.0) {
39:             volume -= 0.1;
40:         }
41: 
42:         bgm.setGain(volume);
43:         bgm.play();
44:     }
45:     return 0;
46: }
\end{verbatim}
\end{small}
\end{itembox}
\subsection*{解説}
\begin{itemize}
 \item FKでは音を出力するための入力として「wav形式」と「ogg形式」を
	サポートしています。wav形式はほとんどの音を扱うソフトウェアで
	利用可能な形式ですが、サイズが大きいという欠点があります。
	ogg形式は mp3 や aac 等と同じくファイルサイズが小さな形式です。
	サポートされているソフトウェアが多いとは言えませんが、
	ライセンスや特許に縛られずに利用できるという利点があり、
	フリーのプログラムでは広く利用されています。
	ogg 形式で音声を作成できるソフトウェアとしては、
	多機能なものとして
	「XRECODE\footnote{http://www.xrecode.com/}」、
	国産として
	「SoundEngine\footnote{
	http://soundengine.jp/software/soundengine/}」
	などがあります。

 \item 音を使うプログラムでは、先に記述したセットアップの他に、
	2行目にあるように\\
	\verb+#include <FK/Audio.h>+
	という記述を追加しておいて下さい。

 \item 10行目にあるのが、BGMの読み込みや再生を行うための変数です。
	基本的に BGM は wav ファイルですと巨大になりますので、
	ogg ファイルのみを用います。

 \item 13行目に ogg ファイルを指定することで再生の準備を行います。
	ファイルがない場合や、正常に入力できなかった場合は
	エラー用のダイアログを 14 行目で開くようにしています。

 \item 16行目は、リピート再生を行うための設定です。
	このように「setLoopMode()」関数で true を引数として
	入力しておくと、曲の終端に来てから最初に戻って
	再生しなおします。false にすると、終端で再生を停止します。
	デフォルトでは false になっています。

 \item 31行目で音量を制御するための変数「volume」の値を
	0.5 に設定していますが、この値はZ,Xキーを押すことで
	上下できるようなプログラムになっています。
	35 行目で「getKeyStatus()」関数を用いてキーの状態を
	取得していますが、これまでの
	「FK\_SW\_PRESS」ではなく「FK\_SW\_DOWN」となっています。
	これは、「押された瞬間」を意味します。
	「FK\_SW\_PRESS」の場合、押されている間はずっと true となります。
	そのため、volume の値はわずかな時間で大きく変動してしまいます。
	「FK\_SW\_DOWN」の場合、押された瞬間しか検知しないため、
	いわゆる「押さえっぱなし」の状況のときは false となります。

 \item 42行目の「setGain()」関数は音量を設定するための関数です。
	1 で最大、0 で無音となります。

 \item 43 行目の「play()」関数ですが、これによって再生が行われます。
	これがゲーム中でサウンドを扱う特殊な面なのですが、
	この「play()」関数は再生している間、何度も呼ぶ必要があります。
	以下、少し詳しく説明します。
	(理解しなくても利用には差し支えありません。)

	音データは全てを読み込んでしまうと膨大なメモリを消費します。
	そのため、音データを「チャンク」と呼ばれる細かなデータに分け、
	チャンクを少しずつファイルから読み込み、システムに流すという方法で
	内部では再生を行っています。この「play()」という関数は、
	内部ではチャンクをファイルから取り出し、システムの状況を検査して
	チャンクを新たに受け付けられる状況であれば流すという処理を
	行っているのです。そのため、かなり頻繁にこの関数を呼ばないと
	途中で再生が止まってしまうのです。

\end{itemize}
\section{サウンドエフェクト(SE)の制御} \label{sec:10-wav}
\begin{itembox}[l]{WAVファイルの入力}
\begin{footnotesize}
\begin{verbatim}
 1: #include <FK/FK.h>
 2: #include <FK/Audio.h>
 3: 
 4: int main(int, char **)
 5: {
 6:     fk_AppWindow        window;
 7:     fk_Block            block(1.0, 1.0, 1.0);
 8:     fk_Model            blockModel;
 9:     fk_Vector           origin(0.0, 0.0, 0.0);
10:     fk_AudioWavBuffer   se[2];
11:     bool                seFlag[2];
12:     int                 i;
13: 
14:     if(se[0].open("MIDTOM2.wav") == false) {
15:         fl_alert("Sound(BD) Open Error");
16:     }
17:     if(se[1].open("SDCRKRM.wav") == false) {
18:         fl_alert("Sound(SD) Open Error");
19:     }
20: 
21:     blockModel.setShape(&block);
22:     blockModel.glMoveTo(3.0, 3.0, 0.0);
23:     blockModel.setMaterial(Yellow);
24:     window.entry(&blockModel);
25: 
26:     window.setCameraPos(0.0, 1.0, 20.0);
27:     window.setCameraFocus(0.0, 1.0, 0.0);
28: 
29:     window.setSize(800, 600);
30:     window.setBGColor(0.6, 0.7, 0.8);
31:     window.open();
32:     window.showGuide(FK_GRID_XZ);
33: 
34:     for(i = 0; i < 2; i++) {
35:         seFlag[i] = false;
36:         se[i].setGain(1.0);
37:     }
38: 
39:     while(window.update() == true) {
40:         blockModel.glRotateWithVec(origin, fk_Y, FK_PI/360.0);
41: 
42:         if(window.getKeyStatus('Z') == FK_SW_DOWN) {
43:             seFlag[0] = true;
44:             se[0].seek(0.0);
45:         }
46: 
47:         if(window.getKeyStatus('X') == FK_SW_DOWN) {
48:             seFlag[1] = true;
49:             se[1].seek(0.0);
50:         }
51: 
52:         for(i = 0; i < 2; i++) {
53:             if(seFlag[i] == true) {
54:                 seFlag[i] = se[i].play();
55:             }
56:         }
57:     }
58:     return 0;
59: }
\end{verbatim}
\end{footnotesize}
\end{itembox}
\subsection*{解説}
\begin{itemize}
 \item このプログラムは、先ほどの BGM 再生よりもやや高度な処理として
	サウンドエフェクト(SE)を制御してみます。
	SEは、打撃音やジャンプの音など、瞬間的〜数秒程度で再生が終わる
	音素材のことです。SEはBGMと違って特定条件下で何度も再生を
	行うことが多いので、やや扱いが複雑となります。

 \item 10行目の「fk\_AudioWavBuffer」は SE 用の音素材の読み込みや
	再生を行うための型で、wav形式に対応しています。
	もし ogg形式のファイルを読み込みたい場合は
	「fk\_AudioOggBuffer」を使用して下さい。
	型名(クラス名)が異なる以外は、使い方は同じです。

	なお、前節の「fk\_AudioStream」との違いですが、\\
	fk\_AudioStream ではチャンクに分けて読み込みを行うのに対し、
	こちらの「fk\_AudioWavBuffer」はファイルのデータを全て
	メモリ上に読み込んでしまいます。
	メモリを多く利用しますが、高速な動作に対応できます。

 \item 14〜19行目でそれぞれ個別に SE 素材となる音ファイルを入力しておきます。

 \item 35行目は、bool 型という「true」か「false」しか格納できない変数
	(の配列)です。36行目は前節のプログラムと同様、音量設定です。

 \item 先に52〜56行目を説明します。54行目にあるように、
	「play()」関数で音を再生します。今回はリピート再生を無効として
	いますので、終端で再生が終わります。「play()」関数は
	音が終端にきた場合には「false」を返すようになっています。
	結果として、while でのループのたびに play() が実行され、
	再生が終わってない場合は\verb+seFlag[i]+ に true が入っているため
	次回のループでも再生され、再生が終わったら
	\verb+seFlag[i]+ に false が入るので、54 行目は実行されずに
	再生もなされないという仕組みです。

 \item 44,49行目にある「seek()」という関数は、再生箇所を変更する関数で、
	単位は「秒」になっています。今回は「0」を指定することで、
	最初まで巻き戻しを行っていることになります。
	このとき、もし \verb+seFlag[0]+ や \verb+seFlag[1]+ が
	false になっていた場合、true に変更して再び再生が
	行われるようにしています。

	この「seek()」関数による指定ですが、経験則上やや不正確です。
	実際に指定した箇所からピンポイントに再生を行いたい場合は、
	数値をプログラム上で微調整して試行錯誤を行う必要があります。

\end{itemize}

\section{音制御クラスの色々な機能}
これまでにサンプルプログラム中で紹介した機能の他に、
fk\_AudioStream, fk\_AudioWavBuffer, fk\_AudioOggBuffer には
制御を行う関数があります。ここではそれを紹介しておきます。

\begin{description}
 \item[pause()]~ \\
	BGMやSEの再生を停止します。その後 play() を実行すると、
	止めた箇所から再び再生が始まります。

 \item[tell()]~ \\
	現時点での再生位置を取得しますが、やや不正確です。
	以下の様な記述をします。
	\begin{screen}
	\begin{verbatim}
	fk_AudioStream    bgm;
	double            playTime;
	              :
	              :
	playTime = bgm.tell();
	\end{verbatim}
	\end{screen}
\end{description}
