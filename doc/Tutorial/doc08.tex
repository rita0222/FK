\chapter{テクスチャマッピングと画面上への文字表示}
\section{テクスチャマッピング} \label{sec:08-texture}
\begin{itembox}[l]{テクスチャ画像表示}
\begin{small}
\begin{verbatim}
 1: #include <FK/FK.h>
 2: 
 3: int main(int, char **)
 4: {
 5:     fk_AppWindow    window;
 6:     fk_RectTexture  texture;
 7:     fk_Block        block(1.0, 1.0, 1.0);
 8:     fk_Model        texModel, blockModel;
 9:     fk_Vector       origin(0.0, 0.0, 0.0);
10: 
11:     texture.readPNG("Chinmi.png");
12:     texture.setTextureSize(7.0, 5.0);
13:     texModel.setShape(&texture);
14:     texModel.setMaterial(White);
15:     texModel.glMoveTo(0.0, 5.0, 0.0);
16:     window.entry(&texModel);
17: 
18:     blockModel.setShape(&block);
19:     blockModel.glMoveTo(2.0, 6.0, 0.0);
20:     blockModel.setMaterial(Yellow);
21:     window.entry(&blockModel);
22: 
23:     window.setCameraPos(0.0, 5.0, 20.0);
24:     window.setCameraFocus(0.0, 5.0, 0.0);
25:     window.setSize(800, 600);
26:     window.setBGColor(0.6, 0.7, 0.8);
27:     window.open();
28:     window.showGuide(FK_GRID_XZ);
29: 
30:     while(window.update() == true) {
31:         texModel.glRotateWithVec(origin, fk_Y, FK_PI/360.0);
32:     }
33:     return 0;
34: }
\end{verbatim}
\end{small}
\end{itembox}
\subsection*{解説}
\begin{itemize}
 \item 3Dグラフィックスの重要な技術に「\textbf{テクスチャマッピング}」
	というものがあります。これは画像の全体または一部分を3次元空間内に
	表示するもので、リアルタイムグラフィックス技術の中でも
	最重要と言えるものです。このサンプルプログラムは、
	その最も基本的な利用方法として1枚の画像を空間上に配置しています。
	表示用の画像をあらかじめ授業用ページからダウンロードしておいて下さい。

 \item FKにも、テクスチャマッピングを行う方法は様々なものが用途に合わせて
	準備されていますが、今回使う「fk\_RectTexture」という型は、
	画像をそのまま一枚の長方形状に扱うためのものです。
	6行目にあるように、まずは変数を宣言しておきます。

 \item 11行目で画像を読み込んでおきます。この画像ファイルは、
	exe ファイルがあるフォルダと同じ場所に置いておく必要があります。
	今回は PNG 形式の画像を入力していますが、
	他に「readJPEG()」という関数で JPEG 形式を入力することや、
	「readBMP()」という関数で BMP 形式の画像を入力することもできます。

 \item 12行目の setTextureSize 関数は、FK 内での画像テクスチャの
	横幅と縦幅をそれぞれ設定します。縦横比(アスペクト比)は、
	必ずしも元画像の縦横比と一致している必要はありません。

 \item 14行目でマテリアルを設定しています。テクスチャ画像の元の色を
	ほぼそのまま再現したい場合は、ここで白に近いマテリアルを設定しておく
	必要があります。

 \item 31行目でテクスチャ画像を回転させていますが、裏側から見ている場合は
	見ることができません。
\end{itemize}
\section{文字列の画面内表示} \label{sec:08-sprite}
\begin{itembox}[l]{スプライト表示}
\begin{small}
\begin{verbatim}
 1: #include <FK/FK.h>
 2: #include <sstream>
 3: 
 4: string IntToString(int argI)
 5: {
 6:     stringstream ss;
 7:     ss << argI;
 8:     return ss.str();
 9: }
10: 
11: int main(int, char **)
12: {
13:     fk_AppWindow    window;
14:     fk_SpriteModel  sprite;
15:     fk_Block        block(1.0, 1.0, 1.0);
16:     fk_Model        model;
17:     fk_Vector       origin(0.0, 0.0, 0.0);
18:     int             count;
19:     std::string     str;
20: 
21:     if(sprite.initFont("rounded-mplus-1m-bold.ttf") == false) {
22:         fl_alert("Font Read Error");
23:     }
24:     sprite.setPositionLT(-280.0, 230.0);
25:     window.entry(&sprite);
26: 
27:     model.setShape(&block);
28:     model.glMoveTo(0.0, 6.0, 0.0);
29:     model.setMaterial(Yellow);
30:     window.entry(&model);
31: 
32:     window.setCameraPos(0.0, 5.0, 20.0);
33:     window.setCameraFocus(0.0, 5.0, 0.0);
34:     window.setSize(800, 600);
35:     window.setBGColor(0.6, 0.7, 0.8);
36:     window.open();
37:     window.showGuide(FK_GRID_XZ);
38: 
39:     count = 0;
40:     while(window.update() == true) {
41:         str = "count = " + IntToString(count);
42:         sprite.drawText(str, true);
43:         model.glRotateWithVec(origin, fk_Y, FK_PI/360.0);
44:         count++;
45:     }
46:     return 0;
47: }
\end{verbatim}
\end{small}
\end{itembox}
\subsection*{解説}
\begin{itemize}
 \item ここでは「スプライト表示」と呼ばれる手法について紹介します。
	これは簡単に言ってしまうと、3次元空間上ではなく
	スクリーンに直接画像を投影することです。
	先ほど、テクスチャマッピングを使って3D上に画像を配置する
	技術を紹介しました。しかし、これを使って字幕などを表示した場合、
	ある物体がその字幕の手前に配置された場合、字幕を遮ってしまいます。
	そこで、他の物体がどんなに近くにあっても画面上に表示される
	画像というのが必要となります。それを「スプライト表示」と言います。

	「スプライト」は本来は旧世代のゲーム機やハードウェアなどで
	ビデオ情報に直接画素値を書き込む技術のことを言いましたが、
	現在はこのような技術はもう用いられていません。
	しかし、今でもそのときの名残で画面に常に表示される画像を
	「スプライト」と呼ぶ慣習があります。

	スプライト表示の中でもとりわけ需要が高いものに「メッセージ表示」
	があります。たとえばゲームでの現在の得点などがそれに当たります。
	今回は、それを行う仕組みを紹介します。

 \item 4〜9行目の解説はまだ現時点でのこの授業の範疇を超えますが、
	これを書いておくことによって「IntToString()」という自作の関数を
	作っており、これによって数値を文字列に変換することができるものです。

 \item 14行目にスプライト専用の型である「fk\_SpriteModel」型の
	変数が宣言されています。

 \item 19行目の「string」型は、C++で標準の文字列を格納するための型です。

 \item 21行目で、スプライトで表示する文字のフォント用ファイルを
	読み込んでいます。今回は「Rounded M+」というフリーのフォントを
	使用します。このフォント用ファイルも授業用ページから
	ダウンロードしておき、exe と同じフォルダにおいて下さい。
	Windows にもフォントファイルは多く搭載されており、
	それを表示に使用することも可能です\footnote{ただし、そのファイルを
	他のPCにコピーすることはライセンス違反となることがあります。}。
	
 \item 24行目は画面上のどこにスプライトを表示するかを指定します。
	これはFKの標準的な座標系ではなく、画面内のピクセル単位で指定します。
	原点は画面の真ん中になります。

 \item 41行目ですが、これは「count =」という文字列と、
	count の値を文字列に変換したものを合わせています。
	たとえば count の値が 500 だった場合は、str の中は
	「count = 500」という文字列が入ります。

 \item 42行目で実際にスプライトの内容を書き換えます。
	第二引数を false にした場合、既に表示されていた文字列を消去せず、
	その後ろに続けて表示していくようになります。

\end{itemize}
\newpage
\section{練習課題} \label{sec:08-q}
\begin{description}
 \myitem 自分で画像を用意し、それをテクスチャマッピング表示せよ。

 \myitem 6枚のテクスチャを使って立方体(サイコロ形状)を作成し、
	X,Y,Zキーでそれぞれ\(x,y,z\)軸を中心に回転するプログラムを作成せよ。

 \myitem 前回の問題7-4のプログラム(7-4が完成できなかった場合は7-2でもよい)
	に対し、現在の車の3次元空間中の位置をスプライト表示するような
	機能を作成せよ。double型の実数を string に変換するには、
	以下のような関数を main 関数の前に追加しておくとよい。
	\begin{screen}
	\begin{verbatim}
	string DoubleToString(double argD)
	{
	    stringstream ss;
	    ss << argD;
	    return ss.str();
	}
	\end{verbatim}
	\end{screen}

\end{description}
