\chapter{物体の消去と干渉判定}
\section{カウンターとモデル消去} \label{sec:04-count}
\begin{itembox}[l]{カウンターとモデル消去}
\begin{small}
\begin{verbatim}
 1: #include <FK/FK.h>
 2: 
 3: int main(int, char **)
 4: {
 5:     fk_AppWindow    window;
 6:     fk_Block        block(10.0, 10.0, 10.0);
 7:     fk_Sphere       sphere(8, 5.0);
 8:     fk_Model        modelA, modelB;
 9:     int             count, div;
10: 
11:     
12:     modelA.setShape(&sphere);
13:     modelA.setMaterial(Green);
14:     modelA.glMoveTo(-20.0, 0.0, 0.0);
15:     window.entry(&modelA);
16: 
17:     modelB.setShape(&block);
18:     modelB.setMaterial(Yellow);
19:     modelB.glMoveTo(20.0, 0.0, 0.0);
20:     window.entry(&modelB);
21: 
22:     window.setSize(800, 600);
23:     window.setBGColor(0.6, 0.7, 0.8);
24:     window.open();
25: 
26:     for(count = 0; window.update() == true; count++) {
27: 
28:         div = count/100;
29:         if(div % 2 == 0) {
30:             window.entry(&modelA);
31:             modelB.setMaterial(Yellow);
32:         } else {
33:             window.remove(modelA);
34:             modelB.setMaterial(Red);
35:         }
36:     }
37: 
38:     return 0;
39: }
\end{verbatim}
\end{small}
\end{itembox}
\subsection*{解説}
\begin{itemize}
 \item 26行目では、アニメーションループで while ではなく for 文を
	用いています。これにより、count という変数が最初は 0 に初期化され、
	ループが1回まわる度に1ずつ増加していきます。このようにしておくことで、
	現在のループが何回目なのかが count に入ることになります。

	このように、なんらかの回数を数えておく変数を「\textbf{カウンター}」と
	呼ぶことがあり、前回紹介した「フラグ」と共によく利用されます。

 \item 28行目は割り算の値を div に代入していますが、\\
	\verb+count/100+ は数学的には整数ではない数値もありえます。
	しかし、div は int 型ですので中に格納される値は整数値となります。
	このように、実際の計算結果が実数であるにも関わらず、
	無理矢理整数型に値を代入することを「\textbf{丸め演算}」と呼びます。
	int 型の丸め演算では、
	小数点以下は基本的に\underline{切り捨て}となります。
	ですので、
	\begin{screen}
	\begin{verbatim}
	    int a = 59;
	    int b = a/10;
	\end{verbatim}
	\end{screen}
	というプログラムは、b には(四捨五入した6ではなく)5が入ります。

 \item 29行目の、パーセント記号は「割った余り」を求めるための記号(演算子)です。
	「\verb+if(div % 2 == 0)+」というのは、div を 2 で割った
	余りが 0、つまり div が偶数のときということになります。
	結果として、count の百の位が偶数ならこの if の条件が真となり、
	奇数なら偽となります。

 \item 33行目の remove 関数は、一度 window に登録したモデルを
	表示要素から外すための関数です。従って、
	div が奇数の場合は球が表示されなくなります。
	再び表示させるには、34 行目にあるように改めて表示したモデルを
	entry することで可能です。
	既に entry されているモデルに対して entry した場合は、
	内部的には二重に表示されるわけではなく、
	実際には特に動作に変化はありません。
	remove を何度も行う場合も同様です。
\end{itemize}

\section{干渉判定} \label{sec:04-col}
\begin{itembox}[l]{球同士の干渉判定}
\begin{small}
\begin{verbatim}
 1: #include <FK/FK.h>
 2: 
 3: int main(int, char **)
 4: {
 5:     fk_AppWindow    window;
 6:     fk_Sphere       sphere(8, 5.0);
 7:     fk_Model        modelA, modelB;
 8:     fk_Vector       posA, posB, vec;
 9:     double          distance;
10: 
11:     modelA.setShape(&sphere);
12:     modelA.setMaterial(Green);
13:     modelA.glMoveTo(-20.0, 2.0, 0.0);
14:     window.entry(&modelA);
15: 
16:     modelB.setShape(&sphere);
17:     modelB.setMaterial(Yellow);
18:     modelB.glMoveTo(0.0, -2.0, 0.0);
19:     window.entry(&modelB);
20: 
21:     window.setSize(800, 600);
22:     window.setBGColor(0.6, 0.7, 0.8);
23:     window.open();
24: 
25:     while(window.update() == true) {
26:         modelA.glTranslate(0.05, 0.0, 0.0);
27: 
28:         posA = modelA.getPosition();
29:         posB = modelB.getPosition();
30:         vec = posB - posA; // vec = ベクトルAB
31:         distance = vec.dist(); // vec の長さ
32: 
33:         if(distance < 10.0) {
34:             modelA.setMaterial(Red);
35:         } else {
36:             modelA.setMaterial(Green);
37:         }
38:     }
39: 
40:     return 0;
41: }
\end{verbatim}
\end{small}
\end{itembox}

\subsection*{解説}
\begin{itemize}
 \item ここでは「\textbf{干渉判定}」を扱います。干渉判定とは、2つの物体に
	重なっている部分(干渉部分)があるかどうかを調べる処理のことです。
	一般的な形状同士の干渉判定はとても高度な理論ですが、
	今回は最も単純な「2個の球体同士の干渉判定」を扱います。

 \item 2個の球 A, B が干渉しているかどうかの判定は以下のようにして
	求めます。球の中心位置同士の距離を \(L\),
	それぞれの半径を \(r_a, r_b\) としたとき、
	\begin{equation}
		L \leqq r_a + r_b
	\end{equation}
	となった場合、2つの球は干渉しています。

 \item 球の中心同士の距離を求めるには、ベクトルを用います。
	2個の球の中心位置ベクトルをそれぞれ \(\bA, \bB\) としたとき、
	その距離は
	\begin{equation}
	|\lvec{\bA\bB}| = |\bB - \bA|
	\end{equation}
	となります。これは実際には
	\begin{equation}
	|\bB - \bA| = \sqrt{(B_x - A_x)^2 + (B_y - A_y)^2 + (B_z - A_z)^2}
	\end{equation}
	となるのですが、このような複雑な式をプログラムで記述しなくても、
	fk\_Vector 型には「dist()」という便利な関数があり、
	そのベクトルの距離を得ることができます。
	31行目が実際に dist 関数を利用している様子です。

 \item 9行目にある「double」というのは、実数を保管しておくための型です。
	int型変数は整数値しか保管できませんでしたが、
	double型の変数は小数点以下についても保管が可能です。
	31行目は、かなり正確な数値が保管されます。

 \item 30行目で fk\_Vector型の変数で引き算を行っていますが、
	これは数値の引き算ではなくベクトルの引き算になります。
	fk\_Vector型は、ベクトル同士の加減算や実数との積を
	通常の数式と同様に記述することができます。

\end{itemize}

\section{練習課題} \label{sec:04-q}
\begin{description}
 \myitem 2個の球を離して配置し、片方を適当なキーで移動操作できるようにする。
	2つの球が干渉したら、止まっている方の球が消去されるような
	プログラムを作成せよ。

 \myitem 2個の球のうち一つをキー操作で移動でき、もう一つを前回の
	問題3-4の動きをするようにする。干渉したら自動的に移動している側の
	球の色が変わるようなプログラムを作成せよ。

 \myitem 小さな球を 横に10個配置し、やや大きめの球を
	キー操作で移動できるようにする。干渉した小球が消去されるような
	プログラムを作成せよ。

 \myitem 小さな球を格子状に\(10\times10 = 100\)個配置し、やや大きめの球を
	キー操作で移動できるようにする。干渉した小球が消去されるような
	プログラムを作成せよ。
\end{description}
