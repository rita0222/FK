\documentclass[a4paper]{jsarticle}

\usepackage{ascmac}
\usepackage{boites}

\textwidth=520pt
\textheight=710pt
\oddsidemargin=-32pt
\evensidemargin=-32pt
\topmargin=-60pt

\begin{document}
\begin{center}
{\Large Fine Kernel ToolKit システム (Windows C++ 版) \\
	セットアップマニュアル} \\ ~ \\
{\Large FineKernel Project} \\
(2019/10/03版)
\end{center}

\section{Visual Studio のバージョンについて}
Windows における Fine Kernel ToolKit システム C++ 版(以下 FK)は 
Visual Studio 2019(以下「VS19」) または Visual Studio 2017 (以下「VS17」)上において、
ほとんど同様の操作で開発を行うことができる。
本マニュアルではこれらのバージョンについてまとめて解説を行っているが、
メニューの項目名などが微妙に異なる場合がある。
その場合は適宜読み替えること。

\section{FK システムのセットアップ}
VS のインストールが完了したら、FK のセットアップを以下の手順で行う。

\begin{enumerate}
\item 以下の URL から、FK ToolKit VisualStudio 用インストーラの項目にある
ファイルをダウンロードする。

\begin{verbatim}
    http://sourceforge.jp/projects/fktoolkit/releases/
\end{verbatim}

VS19 の C++ 版は「FK\_VC19」、
VS17 の C++ 版は「FK\_VC17」から始まるファイル名になっているので、
必要なものを選択すること。

\item インストーラを起動し、指示に従ってインストールする。
標準の設定では VS19 版は ``\verb+C:\FK_VC19+''、
VS17 版は ``\verb+C:\FK_VC17+''、 という名称のフォルダにインストールされる。
インストール後、一度サインアウトか再起動を行う必要がある。

\end{enumerate}

\section{プロジェクトの作成方法}
VS では、アプリケーションを作成する際には「プロジェクト」を作成する必要がある。
開発には単なるプログラムソースだけでなく、
様々な開発用の設定ファイルが必要となるが、
プロジェクトはそれらを全て引っくるめたものだと考えれば良い。

通常、プロジェクトには様々な設定を行う必要があるが、
FK のセットアップが正常に完了した場合は、
「プロジェクトウィザード」を利用することで簡易に設定を済ませることができる。
以下にその手順を述べる。

\subsection{VS19の場合}
\begin{enumerate}
 \item VS19 を起動し、「新しいプロジェクトの作成」を選択する。

 \item 列挙されているテンプレートの中から「FK\_C++\_Template」を選択し「次へ」を押す。
	(大抵の場合、リストの最後に配置されている。)

 \item 「プロジェクト名」に作成したいプロジェクトの名前を入力し,
	「場所」にプロジェクト全体を置きたいフォルダを指定し、「作成」を押す。
	数秒から数十秒程度でプロジェクトが作成される。

 \item 「表示」メニューから「ソリューションエクスプローラー」を選び、
	「ソリューションエクスプローラー」ダイアログを表示しておく。
	この中の「ソースファイル」ある「main.cpp」をダブルクリックすると、
	サンプルコードを編集することができる。

\end{enumerate}

この手順で、プロジェクトの作成と設定が完了する。実際にはプロジェクト自体に加えて、
その上位に位置する「ソリューション」と呼ばれるファイルも同時に作成される。
VS で作成したプロジェクトを開く場合は、ソリューションを通じて行う。

\subsection{VS17の場合}

\begin{enumerate}
 \item VS を起動し、
	メニューから「ファイル」→「新規作成」→「プロジェクト」を選択する。
 \item 表示されたダイアログの左側にある「プロジェクトの種類」ツリーメニューで、
	「Visual C++」という文字列を選択する。もし「Visual C++」が表示されていない場合は、
	「他の言語」というツリーメニューの直下にある場合が多い。
 \item 右側の「テンプレート」から「FK\_C++\_Template」を選択する。
 \item 「名前」というテキストボックス中に適当なプロジェクト名を入力する。
 \item 「場所」に適当な場所を設定しておく。この場所にプロジェクトのフォルダを配置し、
	プログラムソースを保管することになるので、きちんと場所を把握しておく必要がある。
 \item 「ソリューション名」はなんでもよいが、
	VSのプロジェクトやソリューションの概念がよくわからない場合は
	「名前」(プロジェクト名)と同一にしておく。
 \item 上記の全ての入力を終えたら「OK」を押す。
	数秒から数十秒程度でプロジェクトが作成される。
 \item 「表示」メニューから「ソリューションエクスプローラー」を選び、
	「ソリューションエクスプローラー」ダイアログを表示しておく。
	この中の「ソースファイル」ある「main.cpp」をダブルクリックすると、
	サンプルコードを編集することができる。
\end{enumerate}

この手順で、プロジェクトの作成と設定が完了する。実際にはプロジェクト自体に加えて、
その上位に位置する「ソリューション」と呼ばれるファイルも同時に作成される。
VS で作成したプロジェクトを開く場合は、ソリューションを通じて行う。

\section{コンパイル(ビルド)}
ソースファイルが作成できたらコンパイルを行い、実行ファイル(exe)を生成する。
VS においては、コンパイルを初めとした一連の作業のことを「ビルド」と呼ぶ。
ビルドを行うには、メニュー中の「ビルド」→「ソリューションのビルド」を選択する。

エラーが発生した場合はメッセージを確認し、ソースファイルにおけるエラーの場合は
それを修正する必要がある。ヘッダファイルのインクルードや、リンクの時点でエラーが
発生している場合は、これまでの設定に何らかの不備がある可能性が高いので、
もう一度これまでの手順を確認すること。


\section{実行}
ビルドが正常に終了した場合、プロジェクトファイル (.vcxproj) と
同じフォルダ内にある bin フォルダ中に実行ファイルが生成される。
実行するには、メニューの「デバッグ」→「デバッグなしで開始」を選択する。

\subsection{ファイル読み込み処理を行うプログラムを実行する場合}
プロジェクトフォルダの中に「Resources」というフォルダが存在するが、
その中に入れたファイルやフォルダはビルド時に自動的に exe ファイルと
同一の箇所にコピーされる。そのため、プログラム中で入力する
様々なデータファイルは Resources フォルダ中に入れておくとよい。

\subsection{デバッグビルドとリリースビルドについて}
FK を使用したプログラムは、通常時はリリースビルドで実行することを推奨する。
プロジェクト作成後の初期状態ではデバッグビルドになっているが、
この状態ではプログラムの処理速度が大幅に低下することがある。
開発を行っていくうち、複雑なバグなどが発生した場合にはデバッグビルドで
問題を検証するのが有効であるので、状況に応じて切り替えるとよい。

\section{プロジェクトの保存とロード}
プロジェクト全体の状態を保存するには、メニューの「ファイル」→「すべてを保存」を
実行する。これにより、全てのソースファイルとプロジェクトの情報が保存される。

保存したプロジェクトをロードするには、``.sln''を拡張子に持つファイルを開くか、
VS を起動後メニュー中の「ファイル」→ 「ソリューションを開く」を選択し、
``.sln''を拡張子に持つファイルを選択すればよい。

\end{document}
