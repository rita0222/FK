\documentclass[a4j]{jarticle}

\textwidth=480pt
\textheight=680pt
\topmargin=0pt
\headheight=0pt
\headsep=3pt
\oddsidemargin=-20pt
\evensidemargin=-20pt

\begin{document}
\begin{center}
{\Large Fine Kernel ToolKit システム (Visual Studio 2013/2015 版) \\
	セットアップマニュアル} \\ ~ \\
{\Large FineKernel Project} \\
(2015 年 9/1 版)
\end{center}


\section{Visual Studio のバージョンについて}
Fine Kernel ToolKit システム(以下 FK)は 
Visual Studio(以下 VS) 2013 と 2015 上において、
ほとんど同様の操作で開発を行うことができる。
本マニュアルではこれらのバージョンについてまとめて解説を行っているが、
メニューの項目名などが微妙に異なる場合がある。
その場合は適宜読み替えること。

また VS2013/2015 には、
通常の製品版の他に無償で使用できる Community Edition が存在するが、
こちらのバージョンも問題なく使用できる。
VS2013 には Express Edition も存在し、FK システムもこれに対応しているが、
基本的には Community Edition の利用を推奨する。

\section{FK システムのセットアップ}
VS のインストールが完了したら、FK のセットアップを以下の手順で行う。

\begin{enumerate}
\item 以下の URL から、FK ToolKit VisualStudio 用インストーラの項目にある
ファイルをダウンロードする。

\begin{verbatim}
    http://sourceforge.jp/projects/fktoolkit/releases/
\end{verbatim}

2013 版は「FK\_VC13」、2015 版は「FK\_VC15」から
始まるファイル名になっているので、必要なものを選択すること。

\item インストーラを起動し、指示に従ってインストールする。
標準の設定では、2013 版は ``\verb+C:\FK_VC13+''、
2015 版は ``\verb+C:\FK_VC15+'' という名称のフォルダにインストールされる。

セットアップ完了後に、VS への設定情報追加が行われる。
環境によってはこの処理がうまく完了しない場合があるため、
その場合は手動での設定が必要となる。
詳細は本マニュアルの補足の項を参照すること。


\end{enumerate}

\section{プロジェクトの作成方法}
VS では、アプリケーションを作成する際には「プロジェクト」を作成する必要がある。
開発には単なるプログラムソースだけでなく、
様々な開発用の設定ファイルが必要となるが、
プロジェクトはそれらを全て引っくるめたものだと考えれば良い。

通常、プロジェクトには様々な設定を行う必要があるが、
FK のセットアップが正常に完了した場合は、
「プロジェクトウィザード」を利用することで簡易に設定を済ませることができる。
以下にその手順を述べる。

\begin{enumerate}
 \item VS を起動し、
	メニューから「ファイル」→「新規作成」→「プロジェクト」を選択する。
 \item 表示されたダイアログのうち、
	左側の「プロジェクトの種類」というツリーメニューで「Visual C++」という
	ツリーを選択し、さらにその中の「FK」というツリーを選択する。
 \item 右側の「テンプレート」から「FK ToolKit Project」を選択する。
 \item 「プロジェクト名」というテキストボックス中に適当なプロジェクト名を入力する。
 \item 「場所」に適当な場所を設定しておく。この場所にプロジェクトのフォルダを配置し、
	プログラムソースを保管することになるので、きちんと場所を把握しておく必要がある。
 \item 上記の全ての入力を終えたら「OK」を押す。
	すると「FK ToolKit プロジェクトウィザードにようこそ」というダイアログが開くので、
	初期状態で生成したいソースコードの種類を選択して、「完了」ボタンを押す。
 \item	作成したプロジェクトが開いたら、ツールバー内の「Debug」という項目が
 	選択されているプルダウンメニューを「Release」に変更する。
\end{enumerate}

この手順で、プロジェクトの作成と設定が完了する。実際にはプロジェクト自体に加えて、
その上位に位置する「ソリューション」と呼ばれるファイルも同時に作成される。
VS で作成したプロジェクトを開く場合は、ソリューションを通じて行う。


\section{コンパイル(ビルド)}
ソースファイルが作成できたらコンパイルを行い、実行ファイル(exe)を生成する。
VS においては、コンパイルを初めとした一連の作業のことを「ビルド」と呼ぶ。
ビルドを行うには、メニュー中の「ビルド」→「ソリューションのビルド」を選択する。

エラーが発生した場合はメッセージを確認し、ソースファイルにおけるエラーの場合は
それを修正する必要がある。ヘッダファイルのインクルードや、リンクの時点でエラーが
発生している場合は、これまでの設定に何らかの不備がある可能性が高いので、
もう一度これまでの手順を確認すること。


\section{実行}
ビルドが正常に終了した場合、プロジェクトファイル (.vcxproj) と
同じフォルダ内にある bin フォルダ中に実行ファイルが生成される。
実行するには、メニューの「デバッグ」→「デバッグなしで開始」を選択する。

サウンド機能を利用したプログラムの場合、
起動時に DLL が見つからないというメッセージが出て起動できないことがある。
その場合はライブラリインストール先の``\verb+redist\oainst.exe+''を実行し、
OpenAL をシステムにインストールすることで起動できる。

\subsection{ファイル読み込み処理を行うプログラムを実行する場合}
プログラム上から画像や形状データなどを読み込むプログラムを作成した場合、
カレントフォルダはプロジェクトファイルの存在しているフォルダになるため、
パスの設定に注意が必要である。
fk\_System::setcwd()関数をプログラム中で使用すると、カレントフォルダを
exeファイルと同じフォルダに変更することができるので、main()関数の冒頭で
コールしておくと便利である。詳細はクラスリファレンスを参照すること。

\subsection{デバッグビルドとリリースビルドについて}
FK を使用したプログラムは、通常時はリリースビルドで実行することを推奨する。
プロジェクト作成後の初期状態ではデバッグビルドになっているが、
この状態ではプログラムの処理速度が大幅に低下する。
開発を行っていくうち、複雑なバグなどが発生した場合にはデバッグビルドで
問題を検証するのが有効であるので、状況に応じて切り替えるとよい。


\section{プロジェクトの保存とロード}
プロジェクト全体の状態を保存するには、メニューの「ファイル」→「すべてを保存」を
実行する。これにより、全てのソースファイルとプロジェクトの情報が保存される。

保存したプロジェクトをロードするには、``.sln''を拡張子に持つファイルを開くか、
VS を起動後メニュー中の「ファイル」→ 「ソリューションを開く」を選択し、
``.sln''を拡張子に持つファイルを選択すればよい。

\appendix

\section{補足}
以下はプロジェクトの設定に関する補足である。
前節までで一通りの解説は済んでいるので、本節以降は必要に応じて参照すること。

\subsection{環境変数の利用}
FK システムのインストーラーでは、FK のインストール先パスを
環境変数 ``FK\_VC13\_DIR'' または ``FK\_VC15\_DIR'' に記録している。
新規作成したプロジェクトにおいてインストール先を参照する場合、
2015 ならば ``\verb+C:\FK_VC15+'' に相当する部分を ``\$\verb+(FK_VC15_DIR)+'' と表記することで、
他の PC やユーザーの環境でプロジェクトを開いた場合でも、適切にライブラリを参照できるようにしている。
この環境変数はインストール時に自動設定されるが、
インストール時に正常に設定が完了しなかったり、別のパスに FK のフォルダを移動した場合は、
``\verb+vs_wizard\WizReg??.exe+''(?? は 13 または 15)を実行することで、
環境変数とプロジェクトウィザードの再設定を行うことができる。


\subsection{プロパティシートの利用}
FK システムでは Ver.3.0.0 から、FK に必要な Visual C++ のプロジェクト設定をプロパティシートに集約した。
プロパティシートは ``\verb+vs_wizard\FK3_ApplicationProjectSettings.props+'' に配置されており、
プロジェクトウィザードで新規作成したプロジェクトでも、前述した環境変数経由でこのシートを利用している。

設定をプロパティシート化することにはいくつかメリットがあるが、
ユーザーにとって一番便利なのは、将来的に FK が要求する設定が変更になったとしても、
FK のアップデートの際にプロパティシートが更新されることで、
ユーザー利用者側のプロジェクトには修正無しで変更が適用できることが挙げられる。
また、他のシステムと設定の共存が必要な場合に設定箇所が混線しないで済むことや、
手動で新規にプロジェクトを作成する場合の作業を大幅に削減することも期待できる。

既存のプロジェクトに対してプロパティシートを手動でインポートするには、
VS のメニューの「表示」→(環境によっては「その他のウィンドウ」→)「プロパティマネージャー」を開き、
開いたパネル内のプロジェクト名を右クリックして「既存のプロパティシートの追加」をクリックし、
ライブラリインストール先に配置してある .props ファイルを選択すればよい。

\end{document}
