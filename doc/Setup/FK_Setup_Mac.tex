\documentclass[a4paper]{jsarticle}

\usepackage{ascmac}

\textwidth=500pt
\textheight=750pt
\topmargin=0pt
\headheight=-35pt
\headsep=3pt
\oddsidemargin=-30pt
\evensidemargin=-30pt

\begin{document}
\begin{center}
{\LARGE Fine Kernel ToolKit システム (MacOS X 版) \\
	セットアップマニュアル} \\ ~ \\
{\Large FineKernel Project} \\
(2015 年 12/3 版)
\end{center}

\section{対応する Mac OS X のバージョン}
Fine Kernel ToolKit システム(以下「FK」) の Mac 版は、
現在 OS のバージョンが 10.11.1 (El Capitan)
にて動作確認を行っている。これよりも古いバージョンの Mac OS X については
動作の保証はしないものとする。

MacOSX 上で FK を利用するためには、Xcode が必要になるので、
事前にインストールしておくこと。
また、「Command Line Tools」というシステムも必要となるので、
これも予めインストールしておくこと。

\section{UNIX の知識について}
本マニュアルでは、開発に「ターミナル」を用いた UNIX コマンドを
用いていく。そのため、ファイルやディレクトリのコピーや消去といった、
基本的な UNIX コマンドによる操作の知識を前提とする。

\section{FKのダウンロードとインストール}
次に、FK のセットアップを以下の手順で行う。
\begin{enumerate}
\item 以下の URL から、「MacOS X 用インストーラ」の項目にある
	インストーラをダウンロードする。
	\begin{center}
	\begin{tabular}{c}
	\verb+http://osdn.jp/projects/fktoolkit/releases/+
	\end{tabular}
	\end{center}

\item インストーラを起動し、指示に従ってインストールする。
\end{enumerate}

\section{コンパイル手順}
編集したソースコードをコンパイルし実行するには、以下のような手順を取る。
\begin{enumerate}
 \item \verb+/Library/FK_Lib/etc/Makefile.std+ ファイルを
	作業ディレクトリにコピーし、Makefile という名称に変更する。
 \item シェル上で「make」と入力するとコンパイルが始まる。
	エラーメッセージが出た場合は修正後、改めて「make」を入力する。
 \item コンパイルに成功すると「fk\_prog.app」というディレクトリが
	生成され、そこに実行アプリケーションが格納される。
 \item シェル上で「open fk\_prog.app」と入力するか、
	Finder でダブルクリックすれば実行される。
\end{enumerate}
なお、ソースファイルとして認識される拡張子は「cpp」か「cxx」のいずれかである。
複数のソースファイルがある場合は、全体をコンパイル後に全てをリンクして
一つのアプリケーションを生成する。もし main 関数を持っているソースが
複数あった場合はリンクの段階で失敗してしまうので、
一つのアプリケーションには各々の作業ディレクトリを用意しておくこと。

\section{コンパイル環境のカスタマイズ}
コピーした Makefile をテキストエディタで編集することで、
様々な設定を行うことができる。
代表的なものを以下に挙げる。
\subsubsection*{実行アプリケーション名の変更}
最初が「PROGRAM」で始まる行の「fk\_prog」の名称を変更する。
ちなみに、生成後にシェル上で名称を変更しても問題はない。
\subsubsection*{コンパイル時のオプションを追加する}
「EXTRA\_CFLAGS」で始まる行のイコール記号の右にオプション記述を追加する。
デバッグ用のオプションや、コンパイルの時点での警告制御などは、
ここに適切なオプションを記述しておく。
\subsubsection*{リンク時のオプションを追加する}
「EXTRA\_LFLAGS」で始まる行のイコール記号の右にオプション記述を追加する。
リンクするライブラリやフレームワークの追加などは、
ここの適切なオプションを記述しておく。
\subsubsection*{アプリケーションを置くディレクトリの変更}
作業ディレクトリ以外にアプリケーションを置きたい場合は、
まず「TARGET\_DIR」という項目の右側に移動先のディレクトリを記述する。
相対パスや絶対パスも利用できる。make によってアプリケーションを
生成した後、「make move」と入力することでアプリケーションを
そのディレクトリに移動させることができる。
移動先ディレクトリが存在していなかった場合は、自動的に生成も行う。

\section{FKからのデータファイル参照}
FK では、画像ファイルなど様々なデータをプログラムから
参照することができるが、このようなプログラムを MacOS X 上で
適切に動作させるには、以下のような記述をプログラム中の、
実際にデータを読み込む前の部分に記述しておく必要がある。\\ ~ \\
\begin{screen}
\begin{verbatim}
    #ifdef _MACOSX_
        fk_System::setcwd();
    #endif
\end{verbatim}
\end{screen}

\end{document}
