\chapter{色、マテリアル、パレット} \label{chap:material} ~

この章では fk\_Material と呼ばれる立体の色属性を司るクラスの使用法
を述べる。この章に書かれていることは、のちに立体の色属性を設定するために
必要なものとなる。また、複数のマテリアルを管理するための「パレット」
についても解説する。
\section{色の基本 (fk\_Color)}
まず、色の構成に関しての記述から始めよう。光による色の3元色は、
赤、緑、青である。これらの色の組合せによって、ディスプレイで
映し出されるあらゆる色の表現が可能である。

これらの色の組合せを表現するのが fk\_Color クラスの役目である。
fk\_Color クラスは次のように使用する。
\\
\begin{screen}
\begin{verbatim}
        fk_Color col;

        col.init(0.5, 0.6, 0.7);
\end{verbatim}
\end{screen}
~ \\
3つの引数はそれぞれ Red, Green, Blue の値を表し、0 から 1 の
値を代入することができる。つまり、すべてに 1 を代入したときに白色、
すべてに黒を代入したときに黒色を表現することになる。これは、
次のように初期設定によっての代入も可能である。
\\
\begin{screen}
\begin{verbatim}
        fk_Color col(0.5, 0.6, 0.7);
\end{verbatim}
\end{screen}
~ \\
また、次のように個別に代入することも可能である。
\\
\begin{screen}
\begin{verbatim}
        fk_Color col;

        col.setR(0.5);
        col.setG(0.6);
        col.setB(0.7);
\end{verbatim}
\end{screen}
\section{物質のリアルな表現 (fk\_Material)}
ここでは、マテリアルと呼ばれる物質色の表現法と、その設定法を
記述する。前節で述べたような表現では、まだ物質を表現するには
不十分なのである。たとえば蛍光色のような表現、光沢、透明度といった、
非常に細かな設定がなされて初めて物質感を出すことができる。
ここでは、それらの設定法を述べる。しかし、実際に自分の思い通りに
マテリアルの設定が行えるようになるには、ある程度の試行錯誤が
必要となるだろう。

fk\_Material クラスは表\ref{tbl:fkMaterial1}のようなステータスを持っている。

\begin{table}[H]
\caption{fk\_Material の持つステータス}
\label{tbl:fkMaterial1}
\begin{center}
\begin{tabular}{|c|c|c|}
\hline
ステータス名 & 値の型 & 意味 \\ \hline \hline
alpha & float & 透明度 \\ \hline
ambient & fk\_Color & 環境反射係数 \\ \hline
diffuse & fk\_Color & 拡散反射係数 \\ \hline
emission & fk\_Color & 放射光係数 \\ \hline
specular & fk\_Color & 鏡面反射係数 \\ \hline
shininess & float & 鏡面反射のハイライト \\ \hline
\end{tabular}
\end{center}
\end{table}
それぞれに対しての簡単な説明を付随する。

透明度は、文字通り物質の透明度を指し、0.0 のとき完全な透明、1.0 の
ときに完全な不透明を指す。注意しなければならないのは、例えば
ガラスのような透明な物質感を表現したいときは、他のステータスを
黒に近い色に設定しないと、曇りガラスのような表現になってしまうこと
である。従って、立体が持つ透明感はこの値だけではなく、他の色属性も
考慮に入れる必要がある。なお、立体のシーンへの
登録の順序によって透過処理の有無が変わってしまうので、透過処理を
行いたいモデルはできるだけ後に登録する必要がある。これに関する
詳細は第 \ref{chap:scene} 章で再び述べる。また、透過処理を行う場合は
描画そのものが非常に低速になるため、シーンにおいて
透過処理を実際に行うための設定を行う必要もある。これに関しても
第 \ref{chap:scene} 章で述べる。

環境反射係数は、環境光に対しての反射の度合を示すものである。環境光は、
どのような状態にある面にも同様に照らされる(と仮定された)光である。
したがってこの値が高いと、光の当たってない面も光が当たっている面と
同様な色合いを写し出すので、蛍光色に似たような雰囲気になる。逆に、
この値が低いと露骨に光源の効果が出る。したがって、暗い部屋の中に
光源があるような雰囲気が出る。

拡散反射係数は、普通一般にものの「色」と呼ばれているものを指す。
具体的には、光源に当たることによって反映される色のことである。
この色は、光源に対して垂直な角度になったときに最も明るく反映
されるが、一旦面に照射されればすべての方向に均等に散乱するため、
どの視点から見ても同じ明るさを示す。
この値が高いと、物質の色が素直に現れる。この値が低く、ambient や
diffuse や emission の値も低い場合は、その物体は墨のように黒い
ものとなる。diffuse の値が低く、その他の値のうちの幾つかが
高い値を持つとき、変化に富んだ物質感が醸し出される。

放射光係数は、文字通り自身が放射する光の係数を示す。つまり、
あたかも自身が発光しているかのような効果を出す。しかし、
この物体自身は光源ではないので、他の物体の色に影響することはない。
この値の働きは、あくまで自身が発光しているような効果を出すことだけである。
光源の設定に関しては、\ref{chap:model} 章と \ref{chap:scene} 章で
詳しく述べている。

鏡面反射係数は、文字通り反射の色合いを示すものである。この値は、
ある特定の角度範囲からしか反映されない反射の強さを示すものである。
この値が高いと、鏡のように反射が強くなる\footnote{この値を高くしても、
実際の鏡のような効果(他のオブジェクトが反射して映される)があるわけでは
ない。}。この値が高いと、金属やプラスチックのように表面が滑らかな印象を
受ける。逆に低い場合には、紙や石炭のように表面が粗い印象を受ける。

鏡面反射のハイライトは、鏡面反射の反射角度範囲を設定するものである。
この値は、0 から 128 までの値をとり、この値が大きいほどハイライトは
小さくなり、より金属の質感が増す。逆に値を小さくした場合、質感は
プラスチックのようになる。

それぞれの設定の仕方は次のようになっている。
\\
\begin{breakbox}
\begin{verbatim}
        fk_Material  mat;
        fk_Color     amb(0.3, 0.5, 0.8);
        fk_Color     dif(0.2, 0.4, 0.9);
        fk_Color     emi(0.0, 0.5, 0.3);
        fk_Color     spe(1.0, 0.5, 1.0);

        mat.setAlpha(0.5)               // 透明度の設定
        mat.setAmbient(amb);            // 環境反射係数の設定
        mat.setDiffuse(dif);            // 拡散反射係数の設定
        mat.setEmission(emi);           // 放射光係数の設定
        mat.setSpecular(spe);           // 鏡面反射係数の設定
        mat.setShininess(64.0);         // 鏡面反射のハイライトの設定
\end{verbatim}
\end{breakbox}
~ \\
また、fk\_Color を引数にとる関数は次のように直接代入することもできる。
\\
\begin{breakbox}
\begin{verbatim}
        fk_Material  mat;

        mat.setAlpha(0.5);
        mat.setAmbient(0.3, 0.5, 0.8);          // 環境反射係数の設定
        mat.setDiffuse(0.2, 0.4, 0.9);          // 拡散反射係数の設定
        mat.setEmission(0.0, 0.5, 0.3);         // 放射光係数の設定
        mat.setSpecular(1.0, 0.5, 1.0);         // 鏡面反射係数の設定
        mat.setShininess(64.0);                 // 鏡面反射のハイライトの設定
\end{verbatim}
\end{breakbox}
~

この作業は、ディティールを凝る分には非常にいいのであるが、ときには
色つけは簡単に済ませたいという場面もあるだろう。そのようなとき、
逐一値を設定するのは不便である。このとき、非常に簡易に済ませることの
できる手段が2種類用意されている。

最初の手段は、setAmbDiff というメンバ関数を用いることである。これは、
setAmbient メンバ関数や setDiffuse 関数と同様の使用法なのだが、
関数に代入された値を同時に ambient と diffuse の両方の値に同じく
設定する。テストなど、あまり色の質感が関係ない状況ならば、通常はこれだけ
でも十分である。
\\
\begin{screen}
\begin{verbatim}
        fk_Material     mat1, mat2;

        mat1.setAmbDiff(1.0, 1.0, 0.0);         // 黄色いマテリアル
        mat2.setAmbDiff(1.0, 0.0, 1.0);         // マゼンタのマテリアル
\end{verbatim}
\end{screen}
\\
もうひとつの手段は、あらかじめ準備されているマテリアルを使用して
しまうことである。全部で 40 種類あるこれらのマテリアル群は、どれも
グローバルな変数として利用できる。大抵の場合、これで事が足りるだろう。
なお、このマテリアルを羅列した表を付録 A に掲載しておく。参照して、
適宜使用してほしい。
\section{マテリアルパレット (fk\_Palette)} \label{sec:mat_palette}
FK では、1つのモデルに対して単一のマテリアルを設定するのは
容易に行うことができるが、実際には複数のマテリアルを利用したい場合も
よくある。例えば、ある条件を満たす面のみ異なる色で表示したい場合などである。
このようなとき、「マテリアルパレット」と呼ばれる仕組みを利用する。
ここでいう「パレット」とは、水彩や油彩で用いられるときのパレットの
ことであり、同じような役割を果たす。

大別すると、複数のマテリアル利用の実現は以下の 2 ステップで行う。
\begin{enumerate}
 \item 利用するマテリアルを、パレット用変数に保管する。
 \item 形状中の各位相要素(点、線、面)に対し、どのマテリアルを使うのかを
	番号で指定する。
\end{enumerate}
ここでは、まずパレットを準備する段階について述べる。形状中でのマテリアル指定に
ついては、
\ref{subsec:ifs_palette} 節および
\ref{subsec:solid_matid} 節で詳しく述べる。

マテリアルパレットを準備するには、fk\_Palette 型の変数を定義する。
\\
\begin{screen}
\begin{verbatim}
        fk_Palette      palette;
\end{verbatim}
\end{screen}
~ \\
パレットにマテリアルを登録するには、setPalette 関数を用いる。
\\
\begin{screen}
\begin{verbatim}
        fk_Material     mat1, mat2;
        fk_Palette      palette;
                :
        palette.setPalette(&mat1, 1);
        palette.setPalette(&mat2, 2);
\end{verbatim}
\end{screen}
~ \\
setPalette 関数では、1番目の引数に対象となるマテリアルを表す
fk\_Material 型の変数のポインタを、2番目の引数にはマテリアル
を表すための固有の ID を代入する。形状中の各位相要素に対しては、
この ID を指定することになる。例えば、上記例の場合は
マテリアル ID として「1」を指定した位相要素に対し、mat1 に
入っていたマテリアルの色に描画されることになる。

なお、格納されたマテリアルは上書きが可能である。
setPalette の2番目の引数に、以前利用した ID を再び利用した場合、
前に登録されたマテリアルは消去され、新たに登録されたマテリアルに
上書きされる。結果として、描画時には新しいマテリアルが反映されることになる。

fk\_Palette クラスには、他にも以下のようなメンバ関数がある。
\begin{tabbing}
xx \= xxxx \= \kill
\> \textbf{fk\_Material * getMaterial(int id)} \\
	\> \> \begin{tabular}{p{15cm}}
		パレットに格納されている、ID が id であるマテリアルを
		取得する。もし id に相当するマテリアルがなかった場合は
		nullptr を返す。
	\end{tabular} \\ \\

\> \textbf{int getMaterialSize(void)} \\
	\> \> \begin{tabular}{p{15cm}}
		現在格納されているマテリアルの個数を返す。
	\end{tabular} \\ \\

\> \textbf{vector\(<\)fk\_Material\(>\) * getMaterialVector(void)} \\
	\> \> \begin{tabular}{p{15cm}}
		現在格納されているマテリアル全てを、
		STL の vector 配列型で返す。
		なお、ここで得られた配列中を編集してはならない。
	\end{tabular} \\ \\
\end{tabbing}

