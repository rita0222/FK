\chapter{簡易形状表示システム} \label{chap:viewer} ~

\ref{chap:window} 章では、FK システムでの多彩なウィンドウやデバイス制御機能を
紹介したが、中には FK システムで作成した形状を簡単に表示し、
様々な角度から閲覧したいという用途に用いたい利用者もいるであろう。
そのようなユーザは、fk\_AppWindow や fk\_Window によって
閲覧する様々な機能を構築するのはやや手間である。
そこで、FK システムではより簡単に形状を表示する手段として
fk\_ShapeViewer というクラスを提供している。

\section{形状表示}
利用するには、まず fk\_ShapeViewer 型の変数を定義する。\\ ~ \\
\begin{screen}
\begin{verbatim}
        fk_ShapeViewer  viewer;
\end{verbatim}
\end{screen}
\\ ~

この時点で、多くの GUI が付加したウィンドウが生成される。形状は、
\ref{chap:shape} 章で紹介したいずれの種類でも利用できるが、ここでは例として
fk\_Solid 型及び fk\_Sphere 型の変数を準備する。
この変数が表す形状を表示するには、
setShape() メンバ関数を用いる。setShape() メンバ関数は二つの引数を
取り、最初の引数は立体 ID を表す整数、後ろの引数には形状を表す
変数(のアドレス)を入力する。複数の形状を一度に表示する場合は、
ID を変えて入力していくことで実現できる。\\ ~ \\
\begin{breakbox}
\begin{verbatim}
        fk_ShapeViewer  viewer;
        fk_Solid        solid;
        fk_Sphere       sphere;

        viewer.setShape(0, &solid);
        viewer.setShape(1, &sphere);
\end{verbatim}
\end{breakbox}
~ \\

あとは、draw() 関数を呼ぶことで形状が描画される。通常は、次のように
繰り返し描画を行うことになる。\\ ~ \\
\begin{breakbox}
\begin{verbatim}
        while(viewer.draw() == true) {
                // もし形状を変形するならば、ここに処理内容を記述する。
        }
\end{verbatim}
\end{breakbox}
~ \\

もし終了処理 (ウィンドウが閉じられる、「Quit」がメニューで選択されるなど)
が行われた場合、draw() 関数は false を返すので、その時点で while ループを
抜けることになる。形状変形の様子をアニメーション処理したい場合には、
while 文の中に変形処理を記述すればよい。具体的な変形処理のやり方は、
\ref{sec:movevertex} 節及び \ref{sec:sampleviewer} 節に
解説が記述されている。

\section{標準機能}
この fk\_ShapeViewer クラスで生成した形状ブラウザは、次のような機能を
GUI によって制御できる。これらの機能は、何もプログラムを記述することなく
利用することができる。

\begin{itemize}
 \item VRML、STL、DXF 各フォーマットファイル入力機能と、
	VRML ファイル出力機能。
 \item 表示されている画像をファイルに保存。
 \item 面画、線画、点画の各 ON/OFF 及び座標軸描画の ON/OFF。
 \item 光源回転有無の制御。
 \item 面画のマテリアル及び線画、点画での表示色設定。
 \item GUI によるヘディング、ピッチ、ロール角制御及び表示倍率、座標軸サイズの
	制御。
 \item 右左矢印キーによるヘディング角制御。
 \item スペースキーを押すことで表示倍率拡大。また、シフトキーを押しながら
	スペースキーを押すことで表示倍率縮小。
 \item マウスのドラッギングによる形状の平行移動。
\end{itemize}

\section{fk\_ShapeViewer のメンバ関数}
fk\_ShapeViewer クラスの具体的な利用法は、第 \ref{sec:sampleviewer} 節に
記述するが、ここでは fk\_ShapeViewer クラスのメンバ関数一覧を掲載する。
\begin{description}
 \item[\hspace*{0.6cm}bool draw(void)] ~ \\
	実際に描画を行う。

 \item[\hspace*{0.6cm}void setWindowSize(int w, int h)] ~ \\
	描画領域の大きさを \((w, h)\) にする。形状やその他の
	状態は保持される。

 \item[\hspace*{0.6cm}void setShape(int ID, fk\_Shape *shape)] ~ \\
	形状を描画対象として設定する。ID には何か任意の整数値を入れる。
	複数の形状を同時に描画したい場合、ID を変えることで実現できる。
	逆に、前に設定した形状と描画対象を入れ替えたい場合は、
	前に設定した際の ID をそのまま入力することで実現できる。

 \item[\hspace*{0.6cm}void setDrawMode(fk\_DrawMode mode)] ~ \\
	形状の描画モードを選択する。mode に入力できる選択肢は、
	\ref{sec:drawmode} 節で述べているものと同一のものが選択できる。

 \item[\hspace*{0.6cm}void setBGColor(fk\_Color color)]
 \item[\hspace*{0.6cm}void setBGColor(float r, float g, float b)] ~ \\
	形状描画領域の背景色を設定する。

 \item[\hspace*{0.6cm}void setHead(double r)] ~ \\
	カメラのヘッド角を r ラジアンに設定する。

 \item[\hspace*{0.6cm}void setPitch(double r)] ~ \\
	カメラのピッチ角を r ラジアンに設定する。

 \item[\hspace*{0.6cm}void setBank(double r)] ~ \\
	カメラのバンク角を r ラジアンに設定する。

 \item[\hspace*{0.6cm}void setScale(double s)] ~ \\
	表示倍率を s に設定する。

 \item[\hspace*{0.6cm}void setAxisMode(bool mode)] ~ \\
	座標軸について、
	mode が true なら描画有り、false なら無しに設定する。

 \item[\hspace*{0.6cm}void setAxisScale(double l)] ~ \\
	座標軸の長さを l に設定する。

 \item[\hspace*{0.6cm}void setCenter(fk\_Vector pos)]
 \item[\hspace*{0.6cm}void setCenter(double x, double y, double z)] ~ \\
	カメラの注視点を pos または \((x, y, z)\) に設定する。

 \item[\hspace*{0.6cm}void setMaterial(int ID, fk\_Material mat)] ~ \\
	ID が表す形状のマテリアルを mat が表すマテリアルに変更する。

 \item[\hspace*{0.6cm}void setEdgeColor(int ID, fk\_Color col)] ~ \\
	ID が表す形状の稜線色を、col が表す色に変更する。

 \item[\hspace*{0.6cm}void setVertexColor(int ID, fk\_Color col)] ~ \\
	ID が表す形状の頂点色を、col が表す色に変更する。

 \item[\hspace*{0.6cm}void setPosition(int ID, fk\_Vector pos)]
 \item[\hspace*{0.6cm}
	void setPosition(int ID, double x, double y, double z)] ~ \\
	ID が表す形状の位置を変更する。

 \item[\hspace*{0.6cm}void setAngle(int ID, fk\_Angle angle)]
 \item[\hspace*{0.6cm}
	void setAngle(int ID, double h, double p, double b)] ~ \\
	ID が表す形状の姿勢を変更する。h にはヘディング角を、
	p にはピッチ角を、b にはバンク角を入力する。

 \item[\hspace*{0.6cm}void setVec(int ID, fk\_Vector vec)]
 \item[\hspace*{0.6cm} void setVec(int ID, double x, double y, double z)] ~ \\
	ID が表す形状の方向ベクトルを変更する。

 \item[\hspace*{0.6cm}void setUpvec(int ID, fk\_Vector vec)]
 \item[\hspace*{0.6cm}void setUpvec(int ID, double x, double y, double z)] ~ \\
	ID が表す形状のアップベクトルを変更する。

 \item[\hspace*{0.6cm}setBlendStatus(bool mode)] ~ \\
	透過処理について、mode が true であれば有効、
	false であれば無効に設定する。

 \item[\hspace*{0.6cm}void clearModel(void)] ~ \\
	現在の描画登録を全てクリアする。

 \item[\hspace*{0.6cm}fk\_Solid * getShape(int ID)] ~ \\
	ID に設定されている形状データ(のアドレス)を返す。

 \item[\hspace*{0.6cm}fk\_DrawMode getDrawMode(void)] ~ \\
	現在の描画モードを返す。

 \item[\hspace*{0.6cm}double getHead(void)] ~ \\
	現在設定されているカメラのヘッド角をラジアンで返す。

 \item[\hspace*{0.6cm}double getPitch(void)] ~ \\
	現在設定されているカメラのピッチ角をラジアンで返す。

 \item[\hspace*{0.6cm}double getBank(void)] ~ \\
	現在設定されているカメラのバンク角をラジアンで返す。

 \item[\hspace*{0.6cm}double getScale(void)] ~ \\
	現在の表示倍率を返す。

 \item[\hspace*{0.6cm}bool getAxisMode(void)] ~ \\
	現在の座標軸描画の有無を返す。

 \item[\hspace*{0.6cm}double getAxisScale(void)] ~ \\
	現在の座標軸の長さを返す。

 \item[\hspace*{0.6cm}fk\_Vector getCenter(void)] ~ \\
	現在のカメラの注視点を返す。

 \item[\hspace*{0.6cm}fk\_Color getBGColor(void)] ~ \\
	背景色を返す。

 \item[\hspace*{0.6cm}bool getBlendStatus(void)] ~ \\
	現在の透過処理状態を返す。

 \item[\hspace*{0.6cm}void shapeProcess(fk\_Solid *argSolid)] ~ \\
	この関数は、他の関数と違い仮想関数として定義されており、
	派生クラスにて上書きされることを前提としている。この関数は、
	ファイルが入力された時点で必ず呼ばれるものであり、入力した
	形状データが argSolid に引数として入力されるようになっている。
	従って、ファイル入力時に形状データに対して修正や分析
	などを行いたいときには、この関数の中に記述すればよい。

 \item[\hspace*{0.6cm}bool snapImage(const string fileName,
			fk\_ImageType type,
			fk\_SnapProcMode mode)] ~ \\
	現在ウィンドウに表示されている画像を、
	指定した画像形式で fileName にあるファイル名で保存する。
	type は以下の3種類のうちいずれかを指定する。
	\begin{itemize}
	 \item FK\_IMAGE\_BMP (Windows Bitmap 形式)
	 \item FK\_IMAGE\_PNG (PNG 形式)
	 \item FK\_IMAGE\_JPG (JPEG 形式)
	\end{itemize}
	また、mode はどの画像バッファから情報を取得するかを指定する物で、
	以下の 3 種類のいずれかを指定する。
	\begin{itemize}
	 \item FK\_SNAP\_GL\_FRONT (OpenGL フロントバッファ)
	 \item FK\_SNAP\_GL\_BACK (OpenGL バックバッファ)
	 \item FK\_SNAP\_WIN32\_GDI
	 (Windows GDI バッファ、Windows 上でのみ有効)
	 \item FK\_SNAP\_D3D\_WINDOW
	(Direct3D バッファ、Direct3D版のみ有効)
	\end{itemize}

 \item[\hspace*{0.6cm}bool snapImage(fk\_Image *image,
				fk\_SnapProcMode mode)] ~ \\
	現在ウィンドウに表示されている画像を、
	fk\_Image 型の変数に格納する。mode に関しては
	上記のファイル出力の場合と同様である。
	また、mode はデフォルト引数が設定されており、
	省略可能である。省略した場合 mode は FK\_SNAP\_GL\_FRONT となる。

	出力に成功すれば true を、失敗すれば false を返す。
 \item[\hspace*{0.6cm}void setPutStrMode(fk\_PutStrMode mode)]
 \item[\hspace*{0.6cm}fk\_PutStrMode getPutStrMode(void)]
 \item[\hspace*{0.6cm}bool setPutFile(string fileName)]
 \item[\hspace*{0.6cm}void putString(string message)]
 \item[\hspace*{0.6cm}void printf(char *format, ...)]
 \item[\hspace*{0.6cm}void clearBrowser(void)] ~ \\
	これらの関数は、fk\_Window クラスと同様にメッセージ出力機能を
	利用するためのものである。詳細は \ref{subsec:winmessage}節を参照のこと。

\end{description}
